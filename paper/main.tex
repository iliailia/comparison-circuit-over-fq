\documentclass{llncs}

%%%%%%%%%%%%%%%%%%%%%%%%%%%%%%%%%%%%%%%%%%%%% General %%%%%%%%%%%%%%%%%%%%%%%%%%%%%%%%%%%%%%%%%%%%%%
\usepackage[dvipsnames]{xcolor}
\usepackage{array}
\usepackage{nameref}
\usepackage{acronym}
\usepackage{diagbox}
\usepackage{enumitem}
\usepackage{multirow}
\usepackage{booktabs}
\usepackage{makecell}
\usepackage{caption}

%%%%%%%%%%%%%%%%%%%%%%%%%%%%%%%%%%%%%%%%%% Mise en page %%%%%%%%%%%%%%%%%%%%%%%%%%%%%%%%%%%%%%%%%%%%
%\usepackage[left=2cm,right=2cm,top=3cm,bottom=3.2cm]{geometry}
%\usepackage[T1]{fontenc}
%\usepackage[english]{babel}
%\usepackage[latin1]{inputenc}

%%%%%%%%%%%%%%%%%%%%%%%%%%%%%%%%%%%%%%%%%%% References %%%%%%%%%%%%%%%%%%%%%%%%%%%%%%%%%%%%%%%%%%%%%
\usepackage[colorlinks=true,linkcolor=black,filecolor=magenta,urlcolor=black,citecolor=blue]{hyperref}  

%%%%%%%%%%%%%%%%%%%%%%%%%%%%%%%%%%%%%%%%%%%%%% Maths %%%%%%%%%%%%%%%%%%%%%%%%%%%%%%%%%%%%%%%%%%%%%%%
\usepackage{bm}
%\usepackage{amsthm}
\usepackage{amsmath}
\usepackage{amssymb}
\usepackage{amsfonts}
%\usepackage[cmintegrals]{newtxmath}

%%%%%%%%%%%%%%%%%%%%%%%%%%%%%%%%%%%%%%%%%%%%%% TIK'Z %%%%%%%%%%%%%%%%%%%%%%%%%%%%%%%%%%%%%%%%%%%%%%%
\usepackage{tikz}
\usetikzlibrary{calc}
\usepackage{pgfplots}
\pgfplotsset{compat=1.3}

%%%%%%%%%%%%%%%%%%%%%%%%%%%%%%%%%%%%%%%%%%% Algorithmic %%%%%%%%%%%%%%%%%%%%%%%%%%%%%%%%%%%%%%%%%%%%
\usepackage[ruled,vlined,linesnumbered]{algorithm2e}
\usepackage{algorithmicx}
\usepackage[noend]{algpseudocode}

%%%%%%%%%%%%%%%%%%%%%%%%%%%%%%%%%%%%%%%%%%%%% Graphics %%%%%%%%%%%%%%%%%%%%%%%%%%%%%%%%%%%%%%%%%%%%%
\usepackage{url}
\usepackage{subfig}
\usepackage{wrapfig}
\usepackage{graphicx}
\usepackage{pgfplots}
\usepackage{subfloat}
\usepackage{epstopdf}

%%%%%%%%%%%%%%%%%%%%%%%%%%%%%%%%%%%%%%%%%%%%% Commands specific for this paper %%%%%%%%%%%%%%%%%%%%%%%%%%%%%%%%%%%%%%%%%%%%%

%footnotes
%\renewcommand{\thefootnote}{\fnsymbol{footnote}}
\renewcommand{\thefootnote}{\fnsymbol{footnote}}

% Maths
%Vectors
\newcommand{\va}{\mathbf{a}} % arbitrary vectors
\newcommand{\vb}{\mathbf{b}}
\newcommand{\vc}{\mathbf{c}}
\newcommand{\vd}{\mathbf{d}}
\newcommand{\ve}{\mathbf{e}}
\newcommand{\vf}{\mathbf{f}}
\newcommand{\vg}{\mathbf{g}}
\newcommand{\vh}{\mathbf{h}}
\newcommand{\vi}{\mathbf{i}}
\newcommand{\vj}{\mathbf{j}}
\newcommand{\vk}{\mathbf{k}}
\newcommand{\vl}{\mathbf{l}}
\newcommand{\vm}{\mathbf{m}}
\newcommand{\vn}{\mathbf{n}}
\newcommand{\vo}{\mathbf{o}}
\newcommand{\vp}{\mathbf{p}}
\newcommand{\vq}{\mathbf{q}}
\newcommand{\vr}{\mathbf{r}}
\newcommand{\vs}{\mathbf{s}}
\newcommand{\vt}{\mathbf{t}}
\newcommand{\vu}{\mathbf{u}}
% \newcommand{\vv}{\mathbf{v}}
\newcommand{\vw}{\mathbf{w}}
\newcommand{\vx}{\mathbf{x}}
\newcommand{\vy}{\mathbf{y}}
\newcommand{\vz}{\mathbf{z}}
\newcommand{\0}{\mathbf{0}}  % The zero vector
\newcommand{\1}{\mathbf{1}}  % The vector of ones

%Matrices
\newcommand{\mA}{\mathbf{A}} % arbitrary vectors
\newcommand{\mB}{\mathbf{B}}
\newcommand{\mC}{\mathbf{C}}
\newcommand{\mD}{\mathbf{D}}
\newcommand{\mE}{\mathbf{E}}
\newcommand{\mF}{\mathbf{F}}
\newcommand{\mG}{\mathbf{G}}
\newcommand{\mH}{\mathbf{H}}
\newcommand{\mI}{\mathbf{I}} %identity matrix
\newcommand{\mJ}{\mathbf{J}}
\newcommand{\mK}{\mathbf{K}}
\newcommand{\mL}{\mathbf{L}}
\newcommand{\mM}{\mathbf{M}}
\newcommand{\mN}{\mathbf{N}}
\newcommand{\mO}{\mathbf{O}}
\newcommand{\mP}{\mathbf{P}}
\newcommand{\mQ}{\mathbf{Q}}
\newcommand{\mR}{\mathbf{R}}
\newcommand{\mS}{\mathbf{S}}
\newcommand{\mSigma}{\mathbf{\Sigma}}
\newcommand{\mT}{\mathbf{T}}
\newcommand{\mU}{\mathbf{U}}
\newcommand{\mV}{\mathbf{V}}
\newcommand{\mW}{\mathbf{W}}
\newcommand{\mX}{\mathbf{X}}
\newcommand{\mY}{\mathbf{Y}}
\newcommand{\mZ}{\mathbf{Z}}

%functions over vectors and matrices
\newcommand{\proj}{\pi}
\newcommand{\transpose}[1]{#1^{\intercal}} %transpose
\newcommand{\inprod}[2]{\langle #1, #2 \rangle} %inner product
\newcommand{\prodscal}[2]{\left\langle#1, #2\right\rangle} %inner product (Vincent)
\newcommand{\Tr}{\mathrm{Tr}} %trace
\newcommand{\TrOf}[1]{\Tr\!\left(#1\right)}
\newcommand{\No}{\mathrm{No}} %norm

%numerical sets
\newcommand{\Z}{\mathbb{Z}} %integers
\newcommand{\N}{\mathbb{N}} %natural numbers
\newcommand{\Q}{\mathbb{Q}} %rational numbers
\newcommand{\R}{\mathbb{R}} %real numbers
\newcommand{\C}{\mathbb{C}} %complex numbers
\newcommand{\F}{\mathbb{F}} %finite field
\newcommand{\K}{\mathcal{K}} %arbitrary set
\renewcommand{\S}{\mathcal{S}} %arbitrary set

%complex numbers
\newcommand{\conj}[1]{\overline{#1}}

%special constants
\newcommand{\iu}{{i\mkern1mu}} %imaginary i

%ideals
\newcommand{\ideal}[1]{\left\langle #1 \right\rangle}
\newcommand{\idealI}{\mathcal{I}}
\newcommand{\idealJ}{\mathcal{J}}
\newcommand{\idealp}{\mathfrak{p}}

\newcommand{\idealNormOf}[1]{N\!\left(#1\right)}

%sets
\newcommand{\cardinOf}[1]{\left\lvert#1\right\rvert}

%arithmetic functions
\newcommand{\mult}{{\mathrm{mult}}}
\newcommand{\mul}{{\mathrm{mul}}}
\newcommand{\add}{{\mathrm{add}}}
\newcommand{\erf}{{\mathrm{erf}}}
\newcommand{\erfc}{{\mathrm{erfc}}}
\newcommand{\inv}[1]{{{#1}^{-1}}}

%size
\newcommand{\size}{{\mathrm{size}}}

%probability
\newcommand{\Prob}{{\mathrm{Prob}}} % probability

%parameters of any generalized encryption scheme
\newcommand{\ctxtSpace}{\mathcal{C}} %ciphertext space
\newcommand{\ptxtSpace}{\mathcal{P}} %plaintext space
\newcommand{\msgSpace}{\mathcal{M}} %message space
\newcommand{\secpar}{\lambda} %security parameter

%elements of any encryption scheme
\newcommand{\sk}{{\mathtt{sk}}}
\newcommand{\pk}{{\mathtt{pk}}}
\newcommand{\rlk}{{\mathtt{rlk}}}
\newcommand{\ct}{{\mathtt{ct}}}
\newcommand{\pt}{{\mathtt{pt}}}

%basic encryption operations
\newcommand{\KeyGen}{{\mathtt{KeyGen}}}
\newcommand{\SKGen}{{\mathtt{SecretKeyGen}}}
\newcommand{\PKGen}{{\mathtt{PublicKeyGen}}}
\newcommand{\EVGen}{{\mathtt{EvaluateKeyGen}}}
\newcommand{\RKGen}{{\mathtt{RelinKeyGen}}}
\newcommand{\BKGen}{{\mathtt{BootKeyGen}}}
\newcommand{\Encrypt}{{\mathtt{Encrypt}}}
\newcommand{\Enc}{\mathtt{Enc}}
\newcommand{\Decrypt}{{\mathtt{Decrypt}}}
\newcommand{\Dec}{\mathtt{Dec}}
\newcommand{\Encode}{{\mathtt{Encode}}}
\newcommand{\Decode}{{\mathtt{Decode}}}
\newcommand{\Pack}{{\mathtt{Pack}}}
\newcommand{\Unpack}{{\mathtt{Unpack}}}

%homomorphic encryption operations
\newcommand{\Add}{{\mathtt{Add}}}
\newcommand{\AddPlain}{{\mathtt{AddPlain}}}
\newcommand{\Sub}{{\mathtt{Sub}}}
\newcommand{\SubPlain}{{\mathtt{SubPlain}}}
\newcommand{\BasicMul}{{\mathtt{BasicMul}}}
\newcommand{\Mul}{{\mathtt{Mul}}}
\newcommand{\MulPlain}{{\mathtt{MulPlain}}}
\newcommand{\ModSwitch}{{\mathtt{ModSwitch}}}
\newcommand{\Relin}{{\mathtt{Relin}}}
\newcommand{\Rescale}{{\mathtt{Rescale}}}
\newcommand{\Rotate}{{\mathtt{Rot}}}
\newcommand{\Shift}{{\mathtt{Shift}}}
\newcommand{\Frob}{{\mathtt{Frob}}}
\newcommand{\Eval}{{\mathtt{Evaluate}}}
\newcommand{\Select}{{\mathtt{Select}}}
\newcommand{\Replicate}{{\mathtt{Replicate}}}
\newcommand{\IsNonZero}{{\mathtt{IsNonZero}}}
\newcommand{\Power}{{\mathtt{IsNonZero}}}

%lattices
\newcommand{\lat}{\mathcal{L}}
\newcommand{\mindist}{{\lambda_1}}
\newcommand{\mindistOf}[1]{{\lambda_1}\!\left(#1\right)}
\newcommand{\sucmin}[1]{{\lambda_#1}}
\newcommand{\sucminOf}[2]{{\lambda_#1}\!\left(#2\right)}
\newcommand{\smoothpar}[1]{{\eta_#1}}
\newcommand{\smoothparOf}[2]{{\eta_#1}\!\left(#2\right)}
\newcommand{\duallat}[1]{{#1^*}}

%number theory
\newcommand{\numfield}{\mathcal{K}} %the number field 
\newcommand{\numfieldDim}{n} %the dimension of the number field over the field of rational numbers
\newcommand{\defpoly}{f} %the definining polynomial of the number field
\newcommand{\intring}{\mathcal{R}} %the ring of integers of a number field
\newcommand{\expfactor}{\delta_{\mathcal{R}}} %expansion factor 
\newcommand{\dual}[1]{\ensuremath{#1^\vee}} %the dual of an ideal
\newcommand{\codif}{\dual{\intring}}
\newcommand{\canemb}{\sigma} %canonical embedding
\newcommand{\canembMat}{\boldsymbol{\Sigma}} %linear operator of the canonical embedding
\newcommand{\invCanembMat}{\inv{\canembMat}} %invers of the linear operator of the canonical embedding
\newcommand{\canembImg}{H} %image of the canonical embedding in the complex vector space
\newcommand{\canembImgBasis}{\mB} %basis of the can. embedding image over real numbers
\newcommand{\realembNum}{{s_1}} %number of real embeddings
\newcommand{\cmplxembNum}{{s_2}} %number of non-cojugate complex embeddings
\newcommand{\cyclpoly}[1]{{\Phi_{#1}}} %cyclotomic polynomial
\newcommand{\cyclpolyOrd}{m} %order of a cyclotomic polynomial
\newcommand{\CRT}{\mathsf{CRT}} %Chinese remainder theorem ring isomorphism
\newcommand{\disc}{\mathrm{disc}}
\newcommand{\ord}{{\mathrm{ord}}} %order of an element
\newcommand{\eulerphi}[1]{\phi\left(#1\right)} %Euler totient function
\DeclareMathOperator{\Gal}{Gal} %Galois group

%lattice problems
\acrodef{BDD}{Bounded Distance Decoding}
\acrodef{SIS}{Short Integer Solution}
\acrodef{SVP}{Shortest Vector Problem}
\acrodef{uSVP}{Unique Shortest Vector Problem}
\acrodef{GapSVP}{Decision Shortest Vector Problem}
\acrodef{SIVP}{Shortest Independent Vectors Problem}
\acrodef{DGS}{Discrete Gaussian Sampling}
\acrodef{LWE}{Learning with Errors}
\acrodef{SLWE}{Search Learning with Errors}
\acrodef{DLWE}{Decision Learning with Errors}
\acrodef{S(I)VP}{Shortest (Independent) Vector Problem}

%lattice acronyms
\acrodef{ANF}{Algebraic Normal Form}
\acrodef{HNF}{Hermite Normal Form}

%LWE parameters
\newcommand{\LWEdim}{n}
\newcommand{\LWEmod}{q}

%RLWE parameters
\newcommand{\RLWE}{\textsf{RLWE}}
\newcommand{\SRLWE}{\textsf{S-RLWE}}
\newcommand{\DRLWE}{\textsf{D-RLWE}}
\newcommand{\RLWEbasicRing}{\intring}
\newcommand{\RLWEdim}{n}
\newcommand{\RLWEmod}{\LWEmod}
\newcommand{\RLWEring}{{\RLWEbasicRing_\RLWEmod}}
\newcommand{\SCGLWE}{\text{SCG}\textsf{-LWE}}

%HE names
\acrodef{HE}{Homomorphic Encryption}
\acrodef{FHE}{Fully Homomorphic Encryption}
\acrodef{SHE}{Somewhat Homomorphic Encryption}
\acrodef{LHE}{Leveled Homomorphic Encryption}
\acrodef{BGV}{Brakerski-Gentry-Vaikuntanathan}
\acrodef{FV}{Fan-Vercauteren}
\acrodef{GSW}{Gentry-Sahai-Waters}
\acrodef{HAO}{Hiromasa-Abe-Okamoto}
\acrodef{LTV}{Lopez-Tromer-Vaikuntanathan}
\acrodef{YASHE}{Yet Another Somewhat Homomorphic Encryption}
%\newcommand{\BCIV}{\mathtt{BCIV}}
%\newcommand{\HEAAN}{\mathtt{HEAAN}}
%\newcommand{\TFHE}{\mathtt{TFHE}}

%FHE/SHE names
\newcommand{\BGV}{\mathtt{BGV}}
\newcommand{\HEAAN}{\mathtt{HEAAN}}
\newcommand{\FV}{\mathtt{FV}}
\newcommand{\BCIV}{\mathtt{BCIV}}
\newcommand{\TFHE}{\mathtt{TFHE}}

%FV parameters
\newcommand{\FVringDim}{n}
\newcommand{\FVctxtMod}{\LWEmod}
\newcommand{\FVptxtMod}{t}
\newcommand{\FVDelta}{\Delta}
\newcommand{\FVrelinBase}{w}
\newcommand{\FVrlkNum}{\ell}
\newcommand{\FVctxtSpace}{\RLWEbasicRing_\FVctxtMod^2}
\newcommand{\FVptxtSpace}{\RLWEbasicRing_\FVptxtMod}
\newcommand{\slots}{\ell}

%BGV parameters
\newcommand{\BGVptxtMod}{t}

%HEAAN parameters
\newcommand{\HEAANscale}{\Delta}

%RNS operations
\newcommand{\FastBconv}{{\mathtt{FastBconv}}}

%message
\newcommand{\msg}{\mathsf{msg}}

%norms
\newcommand{\abs}[1]{\left\lvert #1 \right\rvert}
\newcommand{\norm}[1]{\left\lVert #1 \right\rVert}
\newcommand{\canorm}[1]{\left\lVert #1 \right\rVert^{\textup{can}}}
\newcommand{\infnorm}[1]{\left\lvert#1\right\rvert_\infty}

% various functions
\newcommand{\wt}{\texttt{wt}}

%modular operation
\newcommand{\modq}[1]{\left[ #1 \right]_\RLWEmod} %modulo (R)LWE modulus in the symmetric interval
\newcommand{\modx}[2]{\left[ #2 \right]_{#1}} %modulo any modulus in the symmetric interval
\newcommand{\modp}[1]{\left[ #1 \right]_p} %modulo p in the symmetric interval
\newcommand{\modt}[1]{\left[ #1 \right]_\FVptxtMod} %modulo the FV plaintext modulus in the symmetric interval
\newcommand{\rem}[2]{\left| #2 \right|_{#1}} %remainder after division

%rounding operations
\newcommand{\round}[1]{\left\lfloor #1 \right\rceil} %rounding
\newcommand{\floor}[1]{\left\lfloor #1 \right\rfloor} %floor
\newcommand{\ceil}[1]{\left\lceil #1 \right\rceil} %ceiling

%distributions
\newcommand{\rand}{\xleftarrow{\$}} %sampled from (Vincent)
\newcommand{\from}{\leftarrow} %sampled from
\newcommand{\ufrom}{\xleftarrow{\$}} %sampled uniformly random from
\newcommand{\udist}{\mathcal{U}} %uniform distribution
\newcommand{\normdist}{\Gamma} %normal distribution
\newcommand{\dgaussdist}{\mathcal{DG}} %discrete Gaussian distribution
\newcommand{\keydist}{\chi_{\text{k}}} %key distribution
\newcommand{\errdist}{\chi_{\text{e}}} %error distribution
\newcommand{\stdev}{\sigma} %standard deviation

%indistinguishability of distributions
\newcommand{\cindist}{\stackrel{\rm c}{\approx}} %computational
\newcommand{\sindist}{\stackrel{\rm s}{\approx}} %statistical

%proof start/end
\newcommand{\bproof}{\noindent {\scshape Proof: }}
\newcommand{\eproof}{\ \mbox{} \hfill $\square$\mbox{}\newline}

%circuits
\newcommand{\circuit}{\mathcal{C}}

%boolean operators
\newcommand{\XOR}{\mathtt{XOR}}
\newcommand{\NAND}{\mathtt{NAND}}
\newcommand{\AND}{\mathtt{AND}}
\newcommand{\EQ}{\mathtt{EQ}}
\newcommand{\LT}{\mathtt{LT}}
\newcommand{\IsNegative}{\mathtt{IsNegative}}
\newcommand{\OR}{\mathtt{OR}}
\newcommand{\NOT}{\mathtt{NOT}}
\newcommand{\MOD}{\mathtt{MOD}}

%complexity classes
\newcommand{\poly}{\mathrm{poly}}

%technical commands
\newcommand{\remove}[1]{}
\newcommand{\alert}[1]{\textcolor{red}{#1}}

%comments
\newcommand{\que}[1]{{\color{red} {(\textbf{question:} #1})\xspace}}
\newcommand{\todo}[1]{\textcolor{red}{(TODO: #1)}}

%cryptographic acronyms
\acrodef{GC}{Garbled Circuit}
\acrodef{DSPR}{Decisional Small Polynomial Ratio}
\acrodef{FHE}{Fully Homomorphic Encryption}

%algorithmic acronyms
\acrodef{SVM}{Support Vector Machine}
\acrodef{LLL}{Lenstra Lenstra Lov{\'a}sz}
\acrodef{GMDH}{Group Method of Data Handling}

%math acronyms
\acrodef{CRT}{Chinese Remainder Theorem}
\acrodef{RNS}{Residue Number System}
\acrodef{NTT}{Number Theoretic Transform}
\acrodef{DFT}{Discrete Fourier Transform}

%computer science acronyms
\acrodef{SIMD}{Single Instruction Multiple Data}
\acrodef{SP}{Streaming Processor}
\acrodef{LSB}{Least Significant Bit}
\acrodef{CPU}{Central Processing Unit}
\acrodef{GPU}{Graphics Processor Unit}
\acrodef{SMP}{Symmetric Multiprocessing}
\acrodef{HPC}{High Performance Computing}
\acrodef{APU}{Accelerated Processing Unit}
\acrodef{FPGA}{Field-Programmable Gate Array}



\newcommand\blankfootnote[1]{%
  \begingroup
  \renewcommand\thefootnote{}\footnote{#1}%
  \addtocounter{footnote}{-1}%
  \endgroup
}
\newcommand*{\skipnumber}[2][1]{%
{\renewcommand*{\alglinenumber}[1]{}\State #2}%
\addtocounter{ALG@line}{-#1}}

%%% Local Variables:
%%% mode: latex
%%% TeX-master: "main"
%%% End:

\newcommand{\fieldcard}{q} %cardinality of a finite field
\newcommand{\fieldchar}{p} %characteristic of a finite field
\newcommand{\princhar}{\chi} %principal character of a finite field
\DeclareMathOperator{\ReLU}{ReLU}

\newif\ifeprint
\eprinttrue

%%%%%%%%%%%%%%%%%%%%%%%%%%%%%%%%%%%%%%%%%%%%%%%%%%%%%%%%%%%%%%%%%%%%%%%%%%%%%%%%%%%%%%%%%%%%%%%%%%%%

\title{Faster homomorphic comparison operations for BGV and BFV}
\author{Ilia Iliashenko\inst{1} \and Vincent Zucca\inst{2,3}}
\institute{imec-COSIC, Dept.\ Electrical Engineering, KU Leuven, Belgium \\ \email{ilia@esat.kuleuven.be}  \and DALI, Université de Perpignan Via Domitia, France, \\ \email{vincent.zucca@univ-perp.fr} \and
    LIRMM, Univ Montpellier, Montpellier, France, \\ \email{vincent.zucca@lirmm.fr}
}

\begin{document}
\maketitle

\begin{abstract} 
  Fully homomorphic encryption (FHE) allows to compute any function on encrypted values.
  However, in practice, there is no universal FHE scheme that is efficient in all possible use cases.
  In this work, we show that FHE schemes suitable for arithmetic circuits (e.g. BGV or BFV) have a similar performance as FHE schemes for non-arithmetic circuits (TFHE) in basic comparison tasks such as less-than, maximum and minimum operations.
  Our implementation of the less-than function in the HElib library is up to 3 times faster than the prior work based on BGV/BFV.
  It allows to compare a pair of 64-bit integers in 11 milliseconds, sort 64 32-bit integers in 19 seconds and find the minimum of 64 32-bit integers in 9.5 seconds on an average laptop without multi-threading.     
\end{abstract}

\section{Introduction}
\label{sec:introduction}
\ac{FHE} gives the ability to perform any kind of computations directly on encrypted data. 
It is therefore a natural candidate for privacy-preserving outsourced storage and computation techniques. 
Since Gentry's breakthrough in 2009~\cite{STOC:Gentry09}, FHE has received a worldwide attention which has resulted in numerous improvements. 
As a result, \ac{FHE} can now be used in practice in many practical scenarios, e.g. genome analysis~\cite{KL15}, energy forecasting~\cite{BCIV17}, image recognition~\cite{BMMP18} and secure messaging~\cite{SP:ACLS18}.\todo{more citations?} 
In addition, FHE is currently going through a standardization process~\cite{HomomorphicEncryptionSecurityStandard}.
  
In practice, homomorphic encryption schemes can be classified into three main categories:
\begin{itemize}
	\item The schemes encrypting their input bit-wise meaning that each bit of the input is encrypted into a different ciphertext. 
	From there, the operations are carried over each bit separately. 
	Examples of such schemes include FHEW \cite{DM15} and TFHE \cite{CGGI16}. 
	These schemes are believed to be the most efficient ones in practice if the metric considered is the \emph{total running time}.
	\item The second category corresponds to word-wise encryption schemes that allow to pack multiple data values into one ciphertext and perform computations on these values in a \ac{SIMD} fashion \cite{SV14}. 
	In particular, encrypted values are packed in different slots such that the operations carried over a single ciphertext are automatically carried over each slot independently. 
	Schemes with these features include BGV~\cite{BGV12} and BFV~\cite{C:Brakerski12,FV12}. 
	Although homomorphic operations in these schemes are less efficient than for bit-wise encryption schemes, their cost per SIMD slot can be better than of the binary-friendly schemes above. 
	We refer to this perfomance metric as the \emph{amortized cost}.
	\item The CKKS scheme~\cite{CKKS17}, which allows to perform computations over approximated numbers, forms the third category. 
	It is similar to the second category in the sense that one can pack several numbers and compute on them in a SIMD fashion.
	The CKKS scheme does not have the algebraic constraints that lower the packing capacity of BGV and BFV. 
	Hence, it is usually possible to pack more elements in a single ciphertext in CKKS, thus resulting the best amortized cost. 
	Unlike previous schemes, CKKS encodes complex, and thus real, numbers natively. 
	However, homomorphic computations are not exact, which means that decrypted results are only valid up to a certain precision. 
\end{itemize}

Each category of schemes is more efficient for a certain application. 
Thus, when comparing the efficiency of different homomorphic schemes, one must take into account a given use case.

It is commonly admitted that schemes of the first category are the most efficient ones for generic applications. 
Since they operate at the bit level, they can compute every logical gate very efficiently. 
The total running time being in this case the sum of the times needed to evaluate each gate of the circuit. 
As a result, to optimize the computations for a given application, the only possibility is to reduce the length of the critical computational path and parallelize the related circuit as much as possible. 
However, as this becomes more and more difficult as the size of the circuit grows, it is possible to optimize only some parts of the circuit by identifying some patterns~\cite{ACS20}.
Another advantage of these schemes is that they have very fast so-called `bootstrapping' algorithms that `refresh' ciphertexts for further computation.
This is very convenient in practice as one can set a standard set of encryption parameters without knowing what function should be computed. 

Schemes of the second category operate naturally on $p$-ary arithmetic circuits, i.e. they are very efficient to evaluate polynomial functions over $\F_p$, for a prime $p$.
However, these schemes become much less efficient when considering other kinds of computations, e.g. comparison operations, step functions. 
To alleviate this problem, one can use tools from number theory to evaluate specific functions with relatively efficient $p$-ary circuits. 
Nonetheless, in general this techniques are too weak to outperform schemes of the first category.
Bootstrapping algorithms of these schemes are quite heavy and usually avoided in practice. 

CKKS, similarly to second category schemes, is very efficient when operating on arithmetic circuits. 
However, unlike other schemes which perform modular arithmetic, it allows to perform computations on complex (and thus real) numbers. 
Although this is an important advantage for many use cases, CKKS lacks simplification tools for evaluation of certain functions due to number-theoretic phenomena as for the second category. 
However, since CKKS usually supports huge packing capacity, it usually presents the best amortized cost.
The bootstrapping algorithm of CKKS is fundamentally different from the above schemes as it refreshes ciphertexts only partially and introduces additional loss of output precision.
Therefore, the CKKS bootstrapping is usually avoided in practice. 

Although \ac{FHE} now offers a relatively efficient alternative for secure computation, some functions remain difficult to evaluate efficiently regardless a scheme. 
Step functions, which are required in many practical applications, form a good example of such functions because of their discontinuous nature. 
The difficulty to evaluate discontinuous functions comes from the hardness to evaluate a quite basic and relatively simple function: the comparison function. 
Although comparison is an elementary operation required in many applications including the famous \emph{Millionaires problem} of Yao \cite{Yao82} or advance machine learning tasks of the iDASH competition\footnote{http://www.humangenomeprivacy.org/2020/index.html}, it remains difficult to evaluate homomorphically.

By now, schemes of the first category look much more suitable for such non-arithmetic tasks, but they are hopelessly inefficient for evaluating arithmetic functions.
Hence, one should resort to heavy conversion algorithms~\cite{JMC:BGGJ20} to leverage the properties of different schemes. 

\subsection{Contributions}
In this work we study the structure of the circuits corresponding to comparison functions for the BGV and BFV schemes. 
For theses schemes, there exists two approaches: either compare two numbers $x$ and $y$ directly by evaluating a bivariate polynomial in $x$ and $y$, or study the sign of the difference $z=x-y$ by evaluating a univariate polynomial in $z$.

By exploiting the structure of these two polynomials, we show that it is possible to evaluate them more efficiently than what was proposed in the state of the art. 
The benefit of our approach results in significant performance enhancement for both methods. 
On the one hand, our bivariate circuit can compare two 64-bit integers with an amortized cost of 21ms, which is a gain of $40\%$ with relation to the best previously reported results of Tan et al.~\cite{TLWRK20} (See Table \ref{table:comparison_circuit_results}). 
On the other hand, our univariate circuit shows even better results with an amortized cost of 11ms for 64-bit numbers -- which is, to the best of our knowledge, more than 3 times faster than previously reported results for this kind of scheme~\cite{TLWRK20}. 
Note that we can compare two 20-bit numbers with an amortized cost of 3ms, which is better by a factor 1.9 than what can be achieved with CKKS-based algorithms and is comparable to TFHE-based implementations (see Table \ref{table:other_he_schemes}).

We also apply our comparison methods to speed up popular computational tasks such as sorting and computing minimum/maximum of an array with $N$ elements. 
For example, for $N=64$, we obtain an amortized cost of 6.5 seconds to sort 8-bit integers and 19.2 seconds for 32-bit integers, which is faster than the prior work by a factor 9 and 2.5 respectively (see Table \ref{table:sorting_circuit_results}). 
For $N=64$, we can also obtain the minimum of 8-bit integers with an amortized running time of 404 ms and of 32-bit integers with an amortized time of 9.57 seconds (see Table \ref{table:minimum_circuit_results}).

\subsection{Related Art}
\label{sec:related-art}
\begin{itemize}
\item \cite{CDSS15}: depth optimized sorting algorithm for HE. Reduce the depth from $\mathcal{O}(l\log^2(N))$ of Batcher network to $\mathcal{O}(\log(N) + \log(l))$ for sorting $N$ $l$-bits integers.

\item \cite{EGNS15}: sorting algorithm for HE. Conclusion: average case in the encrypted domain corresponds to the worst-case in plain domain, better use sorting networks.

\item \cite{CKK15,CKK16} (conference and journal extended version) bit wise comparison using SIMD. Algorithm to compute running products.

\item \cite{KLLW18} equality circuit over non-binary fields. Use the Frobenius automorphism to reduce the depth. 
  
\item \cite{NGEG17}: analysis of bit-wise and digit-wise comparison. Conclusion: bit-wise is more efficient because it has depth $\mathcal{O}(\log(l))$ instead of $\mathcal{O}(l)$ for digit-wise comparisons. Does not use the depth-free Frobenius automorphism\dots

\item \cite{JS19}: bit-wise comparison using SIMD. Conclusion: more efficient than without SIMD. Less interesting than \cite{CKK15} and does not even cite it.

\item \cite{AC:CKKLL19,EPRINT:CheKimKim19}: comparison and min/max functions with CKKS.

\item \cite{AC:CGGI17}: TFHE-based min/max functions. It is faster than we thought. The fastest algorithm to compare two $n$-bit integers takes $170n$ microseconds, which is comparable to our timings.
 
\item \cite{LKN19}: modified shell sort. Make shell sort more efficient for HE from $\mathcal{O}(n^2)$ to $\mathcal{O}(n^{3/2}\sqrt{\alpha+\log\log n})$ with failure probability of $2^{-\alpha}$. Complexity worst than for Batcher even-odd merge sort network.

\item \cite{TLWRK20}: Digit-wise comparison using SIMD. Reduce complexity of digit-wise comparison withfrom $\mathcal{O}(t^{d})$ to $\mathcal{O}(t^{r})$ for $r < d$ by decomposing each element in several digits. Compare numbers up to $64$ bits. Depth smaller than $\log(t-1) + \log(d) + 1$ (same algo than us, probably the work to compare with).

\item \cite{AINA:NGEG17}: the univariate circuit is used in the context of sorting. There is no formal description of the circuit properties and complexity. Neither the decomposition method of~\cite{TLWRK20} or the lexicographic circuit is used.

\item \cite{PoPETS:SFR20}: the univariate circuit is used in the context of top-$k$ selection. As above, there is no formal description of the circuit properties and complexity. Neither the decomposition method of~\cite{TLWRK20} or the lexicographic circuit is used. Their minimum function is based on the comparison table from~\cite{CDSS15}, but its multiplicative complexity is quadratic in the length of an input array. In our case, it is $O(n \log n)$.

\item \cite{JMC:KMNN19}: this work studies polynomial expressions on $\max$, $\argmax$ and other non-arithmetic functions on finite fields. The homomorphic circuit of $\max$ derived from these expressions has a quadratic complexity in $p$. 

\end{itemize}


  
%%% Local Variables:
%%% mode: latex
%%% TeX-master: "main"
%%% End:





%%% Local Variables:
%%% mode: latex
%%% TeX-master: "main"
%%% End:


\section{Background}
\label{sec:background}
\subsection{Notation}

Vectors will be written in column form and denoted by boldface lower-case letters. The vector containing only $1$'s in its coordinates is denoted by $\1$. We write $\0$ for the zero vector. The set of integers $\{\ell,\dots,k\}$ is denoted by $[\ell,k]$.

For a positive integer $t$, let $\texttt{wt}(t)$ be the Hamming weight of its binary expansion. We denote the set of residue classes modulo $p$ by $\Z_p$ and the class representatives of $\Z_p$ are taken from the half-open interval $[-p/2, p/2)$.

\subsection{Cyclotomic fields and Chinese Remainder Theorem}\label{subsec:crt}

Let $m$ be a positive integer and $n = \varphi(m)$ where $\varphi$ is the Euler's totient function. 
Let $\mathcal{K} = \Q(\zeta_{m})$ be the cyclotomic number field constructed by adjoining a primitive $m$-th root of unity $\zeta_{m}\in\C$ to the field of rational numbers. 
The ring of integers of $\mathcal{K}$, denoted by $\intring$, is isomorphic to $\Z[X]/\ideal{\Phi_m(X)}$ where $\Phi_m(X)$ is the $m$-th cyclotomic polynomial. Let $p>1$ be a prime number coprime to $m$, then $\Phi_m(X)$ splits modulo $p$ into $\ell$ irreducible factors of same degree $d$: $\Phi_m(X) = F_1(X)\cdots F_\ell(X) \bmod p$. The degree $d$ is actually the order of $p$ modulo $m$, and $\ell = n/d$. As noticed in \cite{SV14}, the \ac{CRT} states that in this case the following ring isomorphism holds:

\begin{align}\label{eq:crt}
  \intring_p = \Z_p[X]/(\Phi_m(X)) \cong \Z_p[X]/(F_1(X)) \times \ldots \times \Z_p[X]/(F_{\ell}(X))
\end{align}

For each $i \in [1,\ell]$ the quotient ring $\Z_p[X]/(F_i(X))$ is isomorphic to the finite field $\F_{p^d}$. Hence, the isomorphism in~(\ref{eq:crt}) can be rewritten as:
\begin{align*}
  \intring_p \cong \F_{p^d}^\ell.
\end{align*}
We call every copy of $\F_{p^d}$ in the above isomorphism a \emph{slot}. Therefore, every element of $\intring_p$ contains $\ell$ slots, which implies that an array of $\ell$ independent $\F_{p^d}$-elements can be encoded as a unique element of $\intring_p$.
We enumerate the slots according to the enumeration of the polynomials $F_i(X)$'s. Namely, the slot isomorphic to $\Z_p[X]/(F_i(X))$ is referred to as the \emph{$i$th} slot.

Additions and multiplications of $\intring_p$-elements results in the corresponding coefficient-wise operations of their respective slots. In other words, each ring operation on $\intring_p$ is applied to every slot in parallel, this resembles the Single-Instruction Multiple-Data (SIMD) instructions used in parallel computing.

Using multiplication, we can easily define a projection map $\proj_i$ on $\intring_p$ that sends $a \in \intring_p$ encoding slots $(m_0, \dots, m_{\ell-1})$ to $\pi_i(a)$ encoding $(0, \dots, m_i, \dots, 0)$.
In particular, $\proj_i(a) = a g_i$, where $g_i \in \intring_t$ encodes $(0 \dots, 1, \dots, 0)$.
We can generalize this projection for any $I \subseteq \{1,\dots,\ell\}$ to $\proj_I(a) = a g_I$ with $g_I \in \intring_p$ encoding $1$ in the SIMD slots indexed by $I$.\newline

The field $\numfield = \Q(\zeta_{m})$ is a Galois extension and its Galois group $\mathcal{G} = \Gal{(\numfield/\Q)}$ is isomorphic to $\Z_m^\times$  through: $i \mapsto (\sigma_i: X \mapsto X^i)$ where $i \in \Z_m^\times$. The automorphism $\sigma_p$ corresponding to $p$ is called \emph{the Frobenius automorphism} and generates the Galois group $\Gal{(\F_{p^d}/\F_p)}$ of each slot. This means that $\mathcal{F} = <\sigma_p>\subset \mathcal{G}$ partitions the roots of $\Phi_m$  into $\ell$ sets $X_i$ of $d$ elements, each set corresponding to the roots of a factor $F_i$. Therefore the group $\mathcal{H} = \mathcal{G}/\mathcal{F}$ acts transitively on a set of representative $\bar{X}_i$ of each $X_i$ and thus maps a root $\bar{X}_i$ of $F_i$ to a root $\bar{X}_j$ of $F_j$ for $i\neq j$. In other words the elements of $\mathcal{H}$ permute the SIMD slots. However, the order of $\mathcal{H}$ is $n/d = \ell$, which is less than $\ell!$, the number of all possible permutations of the $\ell$ SIMD slots. Nonetheless, it was shown in~\cite{GHS12} that every permutation of SIMD slots can be realized by combination of automorphisms from $\mathcal{H}$, projection maps and additions.


\subsection{Functions over finite fields}
The map defined by $\princhar: x \mapsto x^{p^d-1}$ from $\F_{p^d}$ to the binary set $\{0,1\}$ is called the \emph{principal character}. According to Euler's theorem extended to finite fields, it returns $1$ if $x$ is non-zero and $0$ otherwise. It can thus be used to compute an equality check: $\EQ(x,y) = 1-\chi (x-y)$. Moreover, note that since:
\begin{align}\label{eq:exp_frob}
  a^{p^d-1} = a^{(p-1)(p^{d-1} + \dots + 1)} = \prod_{i=0}^{d-1} (a^{p-1})^{p^i},
\end{align}
$\chi$ can be computed with Frobenius maps and multiplications. Every function from $\F_{p^d}^l$ to $\F_{p^d}$ can be interpolated by a unique polynomial which can be defined with the help of the principal character. 
\begin{lemma}\label{lem:interpolation}
  Every function $f: \F_{p^d}^l \rightarrow \F_{p^d}$ is a polynomial function represented by a unique polynomial $P_f(X_1,\dots,X_{l})$ of degree at most $p^d - 1$ in each variable.
  In particular,
  \begin{align*}
    P_f(X_1,\dots,X_{l}) = \sum_{a_1,\dots,a_l \in \F_{p^d}} f(a_1,\dots,a_l) \prod_{i=1}^{l} \left(1 - \princhar(X_i - a_i)\right)\,.
  \end{align*}
\end{lemma}


% Let $\F_\fieldcard$ be a finite field of characteristic $\fieldchar$.
% We define the multiplicative map $\princhar_\fieldcard: \F_\fieldcard \rightarrow \{0,1\}, x \mapsto x^{\fieldcard-1} \mod \fieldchar$, which is called the \emph{principal character}.
% Due to Euler's theorem, $\princhar_\fieldcard(0) = 0$ and $\princhar_\fieldcard(x) = 1$ for any $x \in \F_\fieldcard^\times$.
% Since
% \begin{align}\label{eq:exp_frob}
% a^{t^d-1} = a^{(t-1)(t^{d-1} + \dots + 1)} = \prod_{i=0}^{d-1} (a^{t-1})^{t^i},
% \end{align}
% the principal character can be realized by Frobenius maps and multiplications.

%%% Local Variables:
%%% mode: latex
%%% TeX-master: "main"
%%% End:


% \section{Homomorpic Encryption}
% \label{sec:HE}
% We are interested in leveled HE schemes that support SIMD operations on their plaintexts.
Such schemes include $\FV$~\cite{FV12} and $\BGV$~\cite{BGV12}, have the ring $\intring_p$ as plaintext space for some $p \geq 2$.
We refer to such schemes as \emph{SIMD-schemes}.
The general framework of such schemes is outlined below.

\subsection{Basic setup}

Let $\lambda$ be the security level of an HE scheme.
Let $L$ be the maximal multiplicative depth of homomorphic circuits we want to evaluate.
Let $d$ be the order of the plaintext modulus $p$ modulo the order $m$ of $\intring$.
Assume that the plaintext space $\intring_p$ has $\slots$ SIMD slots, i.e. $\intring_p \cong \F^\ell_{p^d}$.
% For a vector $\va \in \F^k_{p^d}$, we denote the plaintext encoding of $\va$ by $\pt(\va)$. 
The basic part of a SIMD-scheme consists of key generation, encryption and decryption algorithms.

$\KeyGen(1^\lambda, 1^L) \rightarrow (\sk, \pk)$. Given $\lambda$ and $L$, this function generates the secret key $\sk$ and the public key $\pk$.
Note that the public key contains special key-switching keys that help to transform ciphertexts encrypted under other secret keys back to ciphertexts encrypted under $\sk$. 

$\Encrypt(\pt \in \intring_t, \pk) \rightarrow \ct$. The encryption algorithm takes a plaintext $\pt$ and the public key $\pk$ and outputs a ciphertext $\ct$.

$\Decrypt(\ct, \sk) \rightarrow \pt$. The decryption algorithm takes a ciphertext $\ct$ and the secret key $\sk$ and returns a plaintext $\pt$.
For freshly encrypted ciphertexts, the decryption correctness means that $\Decrypt(\Encrypt(\pt, \pk), \sk) = \pt$. 

\subsection{Arithmetic operations}

Homomorphic arithmetic operations are addition and multiplication.

$\Add(\ct_1, \ct_2) \rightarrow \ct$. The addition algorithm takes two input ciphertexts $\ct_1$ and $\ct_2$ encrypting plaintexts $\pt_1$ and $\pt_2$ respectively.
It outputs a ciphertext $\ct$ that encrypts the sum of these plaintexts in the ring $\intring_p$.
%It implies that homomorphic addition sums respective SIMD slots of $\pt_1$ and $\pt_2$.

$\AddPlain(\ct_1, \pt_2) \rightarrow \ct$. This algorithm takes a ciphertext $\ct_1$ encrypting a plaintext $\pt_1$ and a plaintext $\pt_2$.
It outputs a ciphertext $\ct$ that encrypts $\pt_1 + \pt_2$.
%As for the $\Add$ algorithm, $\AddPlain$ sums respective SIMD slots of $\pt_1$ and $\pt_2$.

$\Mul(\ct_1, \ct_2) \rightarrow \ct$. Given two input ciphertext $\ct_1$ and $\ct_2$ encrypting plaintext $\pt_1$ and $\pt_2$ respectively, the multiplication algorithm outputs a ciphertext $\ct$ that encrypts the plaintext product $\pt_1 \cdot \pt_2$.
%As a result, homomorphic multiplication multiplies respective SIMD slots of $\pt_1$ and $\pt_2$.

$\MulPlain(\ct_1, \pt_2) \rightarrow \ct$. Given a ciphertext $\ct_1$ encrypting plaintext $\pt_1$ and a plaintext $\pt_2$, this algorithm outputs a ciphertext $\ct$ that encrypts the plaintext product $\pt_1 \cdot \pt_2$.
%As a result, homomorphic multiplication multiplies respective SIMD slots of $\pt_1$ and $\pt_2$.

Using the above operations as building blocks, one can design homomorphic subtraction algorithms.

$\Sub(\ct_1, \ct_2) = \Add(\ct_1, \MulPlain(\ct_2, \pt(-\1))) \rightarrow \ct$. The subtraction algorithm returns a ciphertext $\ct$ that encrypts the difference of two plaintext messages $\pt_1 - \pt_2$ encrypted by $\ct_1$ and $\ct_2$, respectively.

$\SubPlain(\ct_1, \pt_2) = \AddPlain(\ct_1, \pt_2 \cdot \pt(-\1)) \rightarrow \ct$. This algorithm returns a ciphertext $\ct$ that encrypts $\pt_1 - \pt_2$ where $\pt_1$ is encrypted by $\ct_1$.
We can also change the order of arguments such that $\SubPlain(\pt_1, \ct_2)$ returns a ciphertext $\ct$ encrypting $\pt_1 - \pt_2$. 

As shown in Section~\ref{subsec:crt}, the projection map $\proj_I$ can select the SIMD slots indexed by a set $I \subseteq \{0,\dots,\slots-1\}$ and set the rest to zero.
This functionality is homomorphically realized by the $\Select$ function.

$\Select(\ct, I) = \MulPlain(\ct, \pt(\1_I)) \rightarrow \ct'$ where $\1_I$ is a vector having $1$'s in the coordinates indexed by a set $I$ and zeros everywhere else.
Given a ciphertext $\ct$ encrypting SIMD slots $\vm = (m_0,m_1,\dots,m_{\slots-1})$ and a set $I$, this function returns a ciphertext $\ct'$ that encrypts $\vm' = (m'_0,\dots,m'_{\slots-1})$ such that $m'_i = m_i$ if $i \in I$ and $m'_i = 0$ otherwise.

\subsection{Special operations}\label{subsec:special_operations}

One can also homomorphically permute the SIMD slots of a given ciphertext and act on them with the Frobenius automorphism. 

$\Rotate(\ct, i) \rightarrow \ct'$ with $i \in [0, \slots-1]$. Given a ciphertext $\ct$ encrypting SIMD slots 
$$\vm = (m_0,m_1,\dots,m_{\slots-1}),$$
the rotation algorithm returns a ciphertext $\ct'$ that encrypts the cyclic shift of $\vm$ by $i$ positions, namely $(m_i,m_{(i+1) \mod \slots},\dots,m_{(i-1) \mod \slots})$.

$\Frob(\ct, i) \rightarrow \ct'$ with $i\in[0,d-1]$. Given a ciphertext $\ct$ encrypting SIMD slots $\vm$ as above, the Frobenius algorithm returns a ciphertext $\ct'$ that encrypts a Frobenius map action on $\vm$, namely $(m^{p^i}_0,m^{p^i}_1,\dots,m^{p^i}_{\slots-1})$.

As discussed in Section~\ref{subsec:crt}, the $\Frob$ and $\Mul$ operations can be combined to compute the principal character $\princhar$, which turns non-zero values of SIMD slots into $1$ and leaves slots with zeros unchanged.

$\IsNonZero(\ct) \rightarrow \ct'$. Given a ciphertext $\ct$ encrypting SIMD slots $\vm = (m_0,m_1,\dots,m_{\slots-1})$, this function returns a ciphertext $\ct'$ that encrypts:
\[
  (\princhar(m_0), \princhar(m_1), \dots, \princhar(m_{\slots-1})).
\]
% Kim et al.~\cite{TDSC:KLLW16} showed that one can employ the Frobenius map to decrease the multiplicative depth of $\IsNonZero$.
Recall that $\princhar(m) = m^{p^d-1} = \prod_{i=0}^{d-1} (m^{p-1})^{p^i}$ as shown in~(\ref{eq:exp_frob}).
The multiplicative depth of $x^{p-1}$ is equal to $\ceil{\log_2 (p-1)}$.
The multiplicative depth of $x^{p^i}$ is zero as it can be done by the $\Frob$ operation.
In total, $d-1$ $\Frob$ operations are needed to compute $\princhar(m)$.
As a result, the total multiplicative depth of $\IsNonZero$ is
\begin{align}\label{eq:nonzero_depth}
  \ceil{\log_2 (p-1)} + \ceil{\log_2 d}.
\end{align}
Using general exponentiation by squaring, $x^{p-1}$ requires $\floor{\log_2 (p-1)} + \wt(p-1) - 1$ field multiplications.
Since $d-1$ field multiplications are needed to compute $\prod_{i=0}^{d-1} (x^{p-1})^{p^i}$, the total number of multiplications to compute $\princhar(m)$ is:
\begin{align}\label{eq:nonzero_width}
  \floor{\log_2 (p-1)} + \wt(p-1) + d - 2.
\end{align} 

\subsection{Cost of homomorphic operations}\label{subsec:cost}
Note that every homomorphic ciphertext contains a special component called \emph{noise} that is removed during decryption.
However, the decryption function can deal only with noise of small enough magnitude; otherwise, this function fails.
This noise bound is defined by encryption parameters in a way that larger parameters result in a larger bound.
The ciphertext noise increases after every homomorphic operation and, therefore, approaches its maximal possible bound.
It implies that to reduce encryption parameters one needs to avoid homomorphic operations that significantly increase the noise.
Therefore, while designing homomorphic circuits, we need to take into account not only the running time of homomorphic operations but also their effect on the noise.  

Table~\ref{table:he_operations} summarizes the running time and the noise cost of the aforementioned homomorphic operations.
Similar to~\cite{C:HalSho14}, we divide the operations into expensive, moderate and cheap.
The expensive operations dominate the cost of a homomorphic circuit.
The moderate operations are less important, but if there are many of them in a circuit, their total cost can dominate the total cost.
The cheap operations are the least important and can be omitted in the cost analysis.

It is worth to note that there are two multiplication functions $\Mul$ (ciphertext-ciphertext multiplication) and $\MulPlain$ (ciphertext-plaintext multiplication).
Since $\Mul$ is much more expensive than $\MulPlain$, the multiplicative depth of a homomorphic circuit is usually calculated with relation to the number of $\Mul$'s.

\begin{table}[t!]
  \centering
  \begin{tabular*}{.45\textwidth}{@{\extracolsep{\fill} } c c c }
    \toprule
    Operation	& Time			& Noise \\
    \midrule
    $\Add$		& cheap			& cheap 	\\
    $\AddPlain$	& cheap			& cheap \\
    $\Mul$		& expensive		& expensive 	\\
    $\MulPlain$	& cheap			& moderate 	\\
    $\Sub$		& cheap			& cheap  \\
    $\SubPlain$	& cheap			& cheap  \\
    $\Select$	& cheap			& moderate \\
    $\Rotate$ 	 & expensive	& moderate \\
    $\Frob$		 & expensive	& cheap \\
    $\IsNonZero$ & expensive    & expensive \\
    \bottomrule
  \end{tabular*}
  \caption{The cost of homomorphic operations with relation to running time and noise growth.}
  \label{table:he_operations}
\end{table}

%%% Local Variables:
%%% mode: latex
%%% TeX-master: "main"
%%% End:


\section{Optimising the comparison circuits over $\mathbb{F}_p$}
\label{sec:comparison-circuit}
As explained in Section \ref{sec:background}, the comparison of elements over $\mathbb{F}_{p^d}$ boils down to their comparison over $\mathbb{F}_p$. In this section we study the structure of the comparison circuit over $\mathbb{F}_p$. As a consequence, in this section, $\S$ is restricted to $[0,p-1]$ in which case we have for any $(x,y)\in\S^2\subseteq \F_p^2$:

$$ \EQ_{\S}(x,y) = 1 - (x-y)^{p-1}. $$

Unfortunately, the comparison function is not as simple and we have to rely on Lagrange interpolation (Lemma 1) to compute it. Yet, there are two different ways to evaluate it: either interpolate $\LT_\S$ as a bivariate function over $S^2$ -- i.e. compute $\LT_\S(x,y)$ -- as done in \cite{TLWRK20}, or transform it as a simple univariate function in $z = x - y$ -- i.e. compute $\LT_S (z, 0)$ -- as done in \cite{AINA:NGEG17} and \cite{PoPETS:SFR20}. However, all these works have evaluated $\LT_\S$ without exploiting its structure. In this Section we study this structure for both the bivariate and univariate cases and we show how to exploit it in order to speed-up their evaluation.


\subsection{Bivariate interpolation of $\LT_\S$.}

  The less-than function can be interpolated using Lemma~\ref{lem:interpolation} and the following truth table
  $$\begin{array}{c|m{1em}m{1em}m{1em}cc}
      < & 0 & 1 & 2 & \cdots & p-1 \\
      \hline
      0 & 0 & 1 & 1 & \cdots & 1 \\
      1 & 0 & 0 & 1 & \cdots & 1 \\
      2 & 0 & 0 & 0 & \cdots & 1 \\
      \vdots & \vdots & \vdots & \vdots & \ddots & \vdots \\
      p-1 & 0 & 0 & 0 & \cdots & 0 \\
    \end{array}$$
    
  In particular it can be interpolated by:
  \begin{align}\label{eq:less_than_function}
    P_{\LT_\S}(x,y) & = \sum_{a = 0}^{p-2} \EQ_\S(x,a)\sum_{b = a+1}^{p-1} \EQ_\S(y,b) \nonumber\\
                &= \sum_{a = 0}^{p-2} (1-\princhar_p(x - a)) \sum_{b = a+1}^{p-1} (1-\princhar_p(y - b)) \nonumber \\
                &= \sum_{a = 0}^{p-2} (1-(x - a)^{p-1}) \sum_{b = a+1}^{p-1} (1-(y - b)^{p-1}).
  \end{align}

  From Eq (\ref{eq:less_than_function}), it appears that $P_{\LT_\S}$ has degree at most $p-1$ in each variable and so, at first sight, it seems to have total degree at most $2p - 2$. It appears that this is not the case:
  \begin{theorem} \label{thm:less_than_total_degree}
    Let $p>2$ be a prime number and $\S = [0,p-1]$, then $P_{\LT_\S}$ is a polynomial over $\F_p$ of the following form:
    \begin{align*}
      P_{\LT_\S}(x,y) = y^{p-1} - \frac{p-1}{2} (xy)^{\frac{p-1}{2}} + \sum_{\substack{i,j>0 \\ i+j \le p}} a_{ij} x^i y^j 
    \end{align*}
    where $a_{ij} = \sum_{a=0}^{p-2} \sum_{b=a+1}^{p-1} a^{p-1-i} b^{p-1-j} \in \F_p$. In particular, it has total degree at most $p$.
  \end{theorem}

  \begin{proof}
    See Appendix \ref{proof_thm_less_than_total_degree}.
  \end{proof}

  Now that we have a better idea of what $P_{\LT_\S}$ looks like we are gonna enumerate some obvious facts which worth being stated:
  \begin{itemize}[label=--]
  \item $P_{\LT_\S} (x, 0) = 0$ thus $y$ divides $P_{\LT_\S} (x, y)$.
  \item $P_{\LT_\S} (x, x) = P_{\LT_\S} (y,y) = 0$ thus $(x - y)$ divides $P_{\LT_\S} (x, y)$.
  \item $P_{\LT_\S} (p-1, y)=0$ thus $x + 1$ divides $P_{\LT_\S} (x, y)$.
  
  \end{itemize}
  Therefore there exists a bivariate polynomial $f$ of total degree $p - 3$ over $\F_p$ such that:
  \begin{equation}
    \label{eq:decomposition-LT}
    P_{\LT_\S} (x, y) = y(x - y)(x + 1)f(x, y) \bmod p.
  \end{equation}

  We will now be interested in the structure of the polynomial $f$.

  \begin{theorem}\label{thm:decomposition-f}
    Let $p$ be an odd prime, for any $z\in\F_p$ we have:
    \begin{equation}
      \label{eq:3}
       f(z,z) = f(z,0) = f(p-1,z).
    \end{equation}
    As a consequence, when $p>3$ there exists a bivariate polynomials $f_1$ of total degree $p-5$ over $\F_p$ such that:
    \begin{align}
      \label{eq:dec-f}
      f(x,y) &= f(x,0) + y(x-y)f_1(x,y) \\
      &= f_0(x) + y(x-y)f_1(x,y).
    \end{align}   
    This decomposition can be applied recursively to $f_1$, so that there exists $(p-1)/2$ polynomials $f_i$ over $\mathbb{F}_p$, $0\leq i \leq (p-3)/2$, such that:

    \begin{align}\label{eq:decomposition-f-final}
      f_0(x,y) = \sum_{i=0}^{(p-3)/2} f_i(x)Y^i,
    \end{align}
    with $Y=y(x-y)$ and $\deg (f_{i}(x)) = p-3 - 2i$.

    % or equivalently a bivariate polynomials $f'_1$ of total degree $p-5$ over $\F_p$ such that:
    % \begin{equation}
    %   \label{eq:dec-f'}
    %   f(x,y) = f(p-1,y) + (x+1)(x-y)f'_1(x,y)
    % \end{equation}    
  \end{theorem}  

  Our proof of Theorem \ref{thm:decomposition-f} is quite long and with no real interest for the purpose of this work. Therefore we defer it to an extended version of this paper. This decomposition can be easily found numerically, we give the formulae of $f$ for small values of $p$ in Appendix \ref{app:decomposition-f}.\newline
  
%   In order to show Eq. (\ref{eq:3}) we will need the following lemma:

%   \begin{lemma}\label{lem:structure-f}
%     Let $p>3$ be an odd prime, $i\in[0,p-2]$ and $f$ be the polynomial defined in Eq. (\ref{eq:decomposition-LT}), then for any $z\in[0,p-1]$ we have:

%     $$ f(i,z) = f(p-1-z,p-1-i) = -\displaystyle\sum_{k=i+1}^{p-1}\frac{1-(z-k)^{p-1}}{(i+1)z(z-i)} \bmod p.$$
%   \end{lemma}

%   \begin{proof}
%     We know that for any $z\in[0,p-1]$ we have:
%     $$\LT_\S(i,z) = \left\{
%       \begin{array}{l}
%         0 ~\text{ if }~ 0\leq z\leq i \\
%         1 ~\text{ otherwise}~
%       \end{array}
%     \right.$$
%     Note that since $i<p-1$, $\LT_\S(i,\cdot)$ is not the zero function. It can be interpolated by the univariate polynomial:
%     $$P_{\LT_\S}(i,z) = \displaystyle\sum_{k=i+1}^{p-1}\left[1-(z-k)^{p-1}\right] \bmod p. $$
    
%     Similarly:
%     \begin{align*}
%       P_{\LT_\S}(x,p-1-i) &= \displaystyle\sum_{k=0}^{p-2}\left[1-(x-k)^{p-1}\right]\displaystyle\sum_{l=k+1}^{p-1}\left[1-(p-1-i-l)^{p-1}\right] \\
%                           &= \displaystyle\sum_{k=0}^{p-2-i}\left[1-(x-k)^{p-1}\right]  \bmod p
%     \end{align*}
%     So:
%     \begin{align*}
%       P_{\LT_\S}(p-1-z,p-1-i) &= \displaystyle\sum_{k=0}^{p-2-i}\left[1-(p-1-z-k)^{p-1}\right]  \bmod p \\
%                               &= \displaystyle\sum_{k=i+1}^{p-1}\left[1-(k-z)^{p-1}\right]  \bmod p \\
%                               &= \displaystyle\sum_{k=i+1}^{p-1}\left[1-(z-k)^{p-1}\right]  \bmod p \\
%                               &=  P_{\LT_\S}(i,z) \bmod p.
%     \end{align*}
%     Now let $A$ be the bivariate polynomial defined over $\F_p$ by:
%     $$A(x,y) = y(x-y)(x+1).$$
%     We have:
%     \begin{align*}
%       A(p-1-z,p-1-i) &= -(i+1)(-z+i)(-z) \bmod p \\
%                      &= -(i+1)z(z-i) \bmod p\\
%                      &= A(i,z) \bmod p.
%     \end{align*}
%     Again, note that since $i < p-1$, $A(i,z)\neq 0 \bmod p$.
    
%     Starting from Eq. (\ref{eq:decomposition-LT}) we can write:
%     \begin{align*}
%       P_i(z) =& A(i,z)f(i,z) = A(p-1-z,p-1-i)f(p-1-z,p-1-i) \bmod p \\
%       \Leftrightarrow & A(i,z)f(i,z) - A(p-1-z,p-1-i)f(p-1-z,p-1-i) = 0 \bmod p \\
%       \Leftrightarrow & A(i,z)(f(i,z) - f(p-1-z,p-1-i)) = 0 \bmod p
%     \end{align*}
%     Given that $A(i,z)\neq 0 \bmod p$ and that $\F_p[z]$ is an integral domain we obtain what we want:

%     $$ f(i,z) = f(p-1-z,p-1-i) \bmod p. $$

%     The last equality is obtained by dividing $P_{LT_\S}(i,z)$ with $A(i,z)\neq 0 \bmod p$.
%   \end{proof}
%   Now we can start the proof of Theorem \ref{thm:decomposition-f}.
%   \begin{proof}[Theorem \ref{thm:decomposition-f}]
%     Equation (\ref{eq:3}) says that the first column, the last line and the descending diagonal of the table of values of $f$ are equals (cf. Table \ref{tab:values-f}).
%     \begin{table}[h]
%       \centering
%       \begin{tabular}{m{1.5em}|m{1em}m{1em}m{1em}m{1em}m{1em}m{1em}c}
%         \backslashbox{$x$}{$y$} & 0 & 1 & 2 & 3 & 4 & 5 & 6 \\
%         \hline
%         0 & \bf{0} & 6 & 5 & 3 & 3 & 5 & 6 \\
%         1 & \bf{4} & \bf{4} & 5 & 4 & 2 & 4 & 5 \\
%         2 & \bf{2} & 0 & \bf{2} & 3 & 2 & 3 & 2 \\
%         3 & \bf{2} & 0 & 0 & \bf{2} & 3 & 4 & 3 \\
%         4 & \bf{2} & 0 & 0 & 0 & \bf{2} & 5 & 5 \\
%         5 & \bf{4} & 0 & 0 & 0 & 0 & \bf{4} & 6 \\
%         6 & \bf{0} & \bf{4} & \bf{2} & \bf{2} & \bf{2} & \bf{4} & \bf{0} \\
%       \end{tabular}
%       \vspace{1em}\caption{Values $f(x,y)$ for $x,y\in\F_7$.}
%       \label{tab:values-f}
%     \end{table}
    
%     Lemma \ref{lem:structure-f} says that for any $0\leq i< p-1$ the $i$-th row (starting from $y=0$) and the $p-1-i$-th column (starting from $x=p-1$) of the table are equals. Another consequence of Lemma \ref{lem:structure-f} is that the values on any descending diagonal of the table form a palindrome (i.e. the values in the table are symmetric around the axis $x = p-1-y$). This is true in particular for the main descending diagonal $(x=y)$.

%     Now let us show that for any $0\leq i < p-1$ we have $f(i,0) = f(i,i)$. The case $i=0$ is trivial, for $i>0$ we know from Lemma \ref{lem:structure-f} that:
%     $$ f(i,z) = -\displaystyle\sum_{k=i+1}^{p-1}\frac{1-(z-k)^{p-1}}{(i+1)z(z-i)} \bmod p, $$
%     which can be rewritten modulo $p$ as:
%     \begin{align*}
%       f(i,z) & = -\frac{p-1-i - \displaystyle\sum_{k=i+1}^{p-1}(z-k)^{p-1}}{(i+1)z(z-i)} = -\frac{p-1-i - \displaystyle\sum_{k=i+1}^{p-1}\displaystyle\sum_{j=0}^{p-1}z^j k^{p-1-j}}{(i+1)z(z-i)} \\
%              & = -\frac{p-1-i - \displaystyle\sum_{j=0}^{p-1}\left(\displaystyle\sum_{k=i+1}^{p-1} k^{p-1-j}\right)z^j}{(i+1)z(z-i)}  = \frac{\displaystyle\sum_{j=1}^{p-1}\left(\displaystyle\sum_{k=i+1}^{p-1} k^{p-1-j}\right)z^{j-1}}{(i+1)(z-i)}.
%     \end{align*}
%     Similarly:
%     \begin{align*}
%       f(i,z) & = -\frac{p-1-i - \displaystyle\sum_{k=1}^{p-1-i}(z-i-k)^{p-1}}{(i+1)z(z-i)} = -\frac{p-1-i - \displaystyle\sum_{k=1}^{p-1-i}\displaystyle\sum_{j=0}^{p-1}(z-i)^j k^{p-1-j}}{(i+1)z(z-i)} \\
%              & = -\frac{p-1-i - \displaystyle\sum_{j=0}^{p-1}\left(\displaystyle\sum_{k=1}^{p-1-i} k^{p-1-j}\right)(z-i)^j}{(i+1)z(z-i)}  = \frac{\displaystyle\sum_{j=1}^{p-1}\left(\displaystyle\sum_{k=1}^{p-1-i} k^{p-1-j}\right)(z-i)^{j-1}}{(i+1)z}
%     \end{align*}

%     So for any $0 < i < p-1$, from the first expression we obtain:
%     $$f(i,0) = -\frac{\sum_{k=i+1}^{p-1} k^{p-2}}{i(i+1)},$$
%     from the second expression we get:
%     \begin{align*}
%       f(i,i) &= \frac{\sum_{k=1}^{p-1-i} k^{p-2}}{i(i+1)} = \frac{\sum_{k'=i+1}^{p-1} (p-k')^{p-2}}{i(i+1)} = \frac{\sum_{k'=i+1}^{p-1} (-k')^{p-2}}{i(i+1)} \bmod p \\
%       &= -\frac{\sum_{k'=i+1}^{p-1} (k')^{p-2}}{i(i+1)} = f(i,0).
%     \end{align*}

%     At this point we have shown that for $0\leq i<p-1$ we have $f(i,0) = f(i,i)$. 
%     This means that the first column and the descending diagonal are equals up to the last line. 
%     Since the values of the descending diagonal form a palindrome, the same applies to the values of the first column and thus of the last line. More precisely, for any $0\leq i < p-1$ we have:
%     $$f(i,0) = f(i,i) = f(p-1-i, p-1-i) = f(p-1-i, 0) = f(p-1, i)$$

%     At this point the last thing we have to prove to get Eq. (\ref{eq:3}) is that $f(0,0) = f(p-1,0) = f(p-1,p-1) \bmod p$. Lemma \ref{lem:structure-f} already gives us $f(0,0) = f(p-1,p-1)$, so we just have to prove the last one.

%     We have:
%     $$ P_{\LT_\S}(0,z)  = -\sum_{j=1}^{p-1}\left(\sum_{k=1}^{p-1}k^{p-1-j}\right)z^j  = z^{p-1} \bmod p \text{ (cf. Lemma \ref{lem:sum_poly}).} $$

%     Moreover, let $A(x,y) = y(x-y)(x+1)$, we have $A(0,z) = -z^2$. So from Eq.~(\ref{eq:decomposition-LT}) we obtain $f(0,z) = -z^{p-3}$ and so $f(0,0) = f(p-1,p-1) = -1 \bmod p$ if $p = 3$ and $0$ otherwise.

%     To conclude we have to show that $f(p-1,0) = -1 \bmod p$ if $p=3$ and $0$ otherwise. 
%     Since both $A$ and $\LT_\S$ vanish on the first column of the table ($y=0$) and the last line ($x=p-1$) we cannot use their values on these lines to derive the value of $f(p-1,0)$. So this time we will estimate $\LT_\S$ and $A$ on the ascending diagonal. Since $A(p-1-z,z) = z^2(1+2z) \bmod p$ if we want to prove that $f(p-1,0) = 0 \bmod p$ (resp. $1$ if $p=3$) we have to show that the coefficients of degree $0$, $1$ and $2$ of $P_{\LT_\S}(p-1-z,z)$ are equals to $0$, (resp $0, 1$ are equals to $0$ and the coefficient of degree $2$ is equal to $-1$).

%     By definition, $\LT_\S(p-1,0) = P_{LT_\S}(p-1,0) = 0 \bmod p$, so the coefficient of degree $0$ is equal to $0$. Now let us have a look at the formal derivative of $P_{\LT_\S}(p-1-z,z)$. We have: 
%     $$ P_{LT_\S}(p-1-z,z) = \sum_{k=0}^{p-2}\underbrace{[1-(1+z+k)^{p-1}]}_{=g_k(z)}\underbrace{\sum_{l=k+1}^{p-1}[1-(z-l)^{p-1}]}_{=h_k(z)} \bmod p $$
 
%   For $0\leq k \leq p-2$, we have $g_k(0) = 1-(1+k)^{p-1} = 0 \bmod p$, and also $h_k(0) = \sum_{l=k+1}^{p-1}[1-l^{p-1}] = 0 \bmod p$ since $0<l\leq p-1$. 

% Therefore $P'_{LT_\S}(p-1,0) = \sum_{k=0}^{p-2}\left(g_k'(0)h_k(0) + g_k(0)h'_k(0) \right) = 0 \bmod p$ therefore the coefficient of degree $1$ is also equal to $0$. 

% Now let us consider the formal derivative at the second order of $P_{\LT_\S}(p-1-z)$:
% $$P''_{\LT_\S}(p-1-z,z) = \sum_{k=0}^{p-2}\left(g_k''(z)h_k(z) + 2g_k'(z)h'_k(z) + g_k(z)h''_k(z) \right).$$
% Thus in $z=0$ we obtain: 
% \begin{align*}
%   P''_{\LT_\S}(p-1,0) & = 2\sum_{k=0}^{p-2}g_k'(0)h'_k(0) = -2\sum_{k=0}^{p-2}(1+k)^{p-2}\sum_{l=k+1}^{p-1}l^{p-2} \\
%                       &= -2\left(\sum_{k=1}^{(p-1)/2}k^{p-2}\sum_{l=k}^{p-1}l^{p-2} + \sum_{k=(p+1)/2}^{p-1}k^{p-2}\sum_{l=k}^{p-1}l^{p-2}\right) \\
%                       & = -2\left(\sum_{k=1}^{(p-1)/2}k^{p-2}\sum_{l=k}^{p-1}l^{p-2} + \sum_{k=1}^{(p-1)/2}(-k)^{p-2}\sum_{l=p-k}^{p-1}l^{p-2}\right) \\
%                       & = -2\sum_{k=1}^{(p-1)/2}k^{p-2}\left(\sum_{l=k}^{p-1}l^{p-2} -  \sum_{l=p-k}^{p-1}l^{p-2}\right) \\
%                       & = -2\sum_{k=1}^{(p-1)/2}k^{p-2}\sum_{l=k}^{p-1-k}l^{p-2} = -2\sum_{k=1}^{(p-1)/2}k^{2p-4} \\
%                       & = -2\sum_{k=1}^{(p-1)/2}k^{p-3} = -\sum_{k=1}^{(p-1)/2}k^{p-3} - \sum_{k=1}^{(p-1)/2}(-k)^{p-3} \\
%                       & = -\sum_{k=1}^{p-1}k^{p-3}  \bmod p\\
%                       & = 1 ~\text{ if } p = 3 ~\text{ and }~ 0 ~\text{ otherwise.}
% \end{align*}
% The coefficient of degree 2 is equal to $P''_{\LT_\S}(p-1,0)/2 = -1 \bmod 3$.
% Thus overall $f(p-1,0) = f(p-1,p-1) = f(0,0)$ and we have proved Eq. (\ref{eq:3}). The rest of the theorem is straightforward:
% \begin{itemize}
% \item $y$ and $(x-y)$ divide $f(x,y)-f(x,0) = f(x,y)-f(x,x) \bmod p$ which implies (\ref{eq:dec-f}). 
% \item $x+1$ and $(x-y)$ divide $f(x,y)-f(p-1,y) = f(x,y)-f(y,y) \bmod p$ which implies (\ref{eq:dec-f'}).
% \end{itemize}
% Note that when $p=3$ $f(x,y) = 1 \bmod p.$
% \end{proof}

\paragraph{Complexity analysis}
In \cite{TLWRK20}, the authors proposed to evaluate $P_{\LT_S}(x,y)$ by evaluating each monomials separately before summing them up. This requires to precompute the powers of $x$ and $y$ up to $p-1$ for a total of $2p-4$ non-scalar multiplications. Then one need another $p-1$ non-scalar multiplications to evaluate each monomial $((\sum_i c_i x^i)y^j)_{j}$ where the $c_i$s are scalars in $\F_p$, before summing them together to get the final result. Overall their evaluation of $P_{\LT_\S}$ requires $3p-5$ non-sclar multiplications. \newline

Following the idea of \cite{TLWRK20} and using the decomposition of $f$ given in Eq. (\ref{eq:decomposition-f-final}) one needs:
\begin{itemize}
\item $2$ multiplications to get $(x+1)Y$ for $p\geq 3$ and only $1$ when $p=2$;
\end{itemize}
and then for $p\geq 5$:
\begin{itemize}
\item $p-4$ multiplications to compute the $x^i$s for $2\leq i \leq p-3$ required to compute the terms $f_i(x,0)$;
\item $(p-5)/2$ multiplications to compute the $Y^i$ for $2\leq i \leq (p-3)/2$;
\item $(p-3)/2$ multiplications to compute the products $f_i(x,0)\cdot Y^i$;
\item $1$ final multiplication to multiply $f$ and $(x+1)Y$.
\end{itemize}

Thus overall, for $p\geq 5$, one needs at most $2p-5$ non scalar multiplications to evaluate $P_{\LT_\S}(x,y)$ homomorphically. Actually by optimising the way of computing the $f_i$ this number can be slightly reduced. For instance for $p=5$ and $7$ it can be done using only $4$ and $6$ multiplications respectively (see Appendix \ref{app:decomposition-f}), hence for a prime $2\leq p \leq 7$, $P_{\LT_\S}(x,y)$ can be computed with only $p-1$ non-scalar multiplications.

Overall, for $p\geq 5$, our method saves $p$ multiplications over the method of \cite{TLWRK20}. However its complexity remains linear in $p$ which is unpractical for performing homomorphic comparisons using large digits (and thus a large $p$).
  
\subsection{Univariate interpolation of $\LT_\S$}
The linear complexity for evaluating the bivariate circuit comes from the fact that it is bivariate. However it is possible to evaluate univariate polynomials of degree $p-1$ in $\mathcal{O}(\sqrt{p})$ non scalar multiplications using \cite{SIAM:PS73}. Therefore it would be interesting to be able to evaluate $\LT_S$ using a univariate polynomial. This can be achieved by computing the difference $x-y$ of the two values to compare and find its sign.
  To compute the sign function (or $\LT_\S(x-y,0)$) using finite field arithmetic, we need to split finite field elements into two classes:  negative ($\F_p^-$) and non-negative ($\F_p^+$).
  In addition, for any $x, y \in \S$ the following property should hold:
  \begin{align*}
    x - y \in 
    \begin{cases}
      \F_p^+ \text{ if } \LT_\S(x,y) = 0, \\
      \F_p^- \text{ if } \LT_\S(x,y) = 1.
    \end{cases}
  \end{align*}   
  Let $\S = [0, (p-1)/2]$ for a prime odd integer $p$.
  Let us split $\F_p$ into $\F_p^+ = [0, (p-1)/2]$ and $\F_p^- = [-(p-1)/2, -1]$.
  Notice that for any $x, y \in \S$, their difference $x - y$ belongs to $\F_p^-$ if and only if $x < y$.
  According to Lemma~\ref{lem:interpolation}, the indicator function of $\F_p^-$ is equal to
  \begin{align*}
    \chi_{\F_p^-}(z) = \sum_{a=-\frac{p-1}{2}}^{-1} 1 - (z - a)^{p-1}.
  \end{align*}
  In other words, this function outputs $1$ if $z$ is negative and $0$ otherwise.
  Combining the above facts the $\LT_\S$ function can be interpolated by the following polynomial.
  \begin{align*}
    Q_{\LT_\S}(x,y) = \sum_{a=-\frac{p-1}{2}}^{-1} 1 - (x - y - a)^{p-1} \bmod p.
  \end{align*}
  The following lemma describes all the coefficients of $Q_{\LT_\S}$.
  \begin{lemma}\label{lem:univariate}
    For an odd prime $p$ and $\S = [0, (p-1)/2]$, the $\LT_\S$ function can be interpolated by the following polynomial over $\F_p$
    \begin{align}\label{eq:univariate_circuit}
      Q_{\LT_\S}(x,y) = \frac{p+1}{2} (x-y)^{p-1} + \sum_{i=1, \text{odd}}^{p-2} c_i (x-y)^i.
    \end{align}
    where $c_i = \sum_{a=1}^{\frac{p-1}{2}} a^{p-1-i}$.
  \end{lemma}
  \begin{proof}
    See Appendix \ref{app:proof-lem-univariate}
  \end{proof}
  \paragraph{Complexity analysis}The above lemma implies that the less-than function can be expressed by a univariate polynomial of degree $p-1$.
  In general, such polynomials are evaluated in $O(p-1)$ steps according to Horner's method.
  However, in homomorphic computation model, multiplication by a scalar coefficient is usually much cheaper then ``non-scalar'' multiplication of expressions containing input variables $x$ and $y$.
  To reduce the number of non-scalar multiplications, we can resort to the Paterson-Stockmeyer algorithm~\cite{SIAM:PS73} that requires $O(\sqrt{2(p-1)})$ such multiplications.
  However, we can improve this complexity by exploiting the fact that the polynomial in~(\ref{eq:univariate_circuit}) has only one coefficient with an even index, the leading one.
  Thus, we can rewrite~(\ref{eq:univariate_circuit}) as follows
  \begin{align*}
    \alpha_{p-1} z^{p-1} + z \sum_{i=0, \text{even}}^{p-3} \alpha_{i+1} z^i =  \alpha_{p-1} z^{p-1} + z g(z^2)
  \end{align*}
  where $z = x - y$, $\alpha_i = \sum_{a=1}^{\frac{p-1}{2}} a^{p-1-i}$ and $g(X)$ is a polynomial of degree $(p-3)/2$.
  To evaluate $g(X)$, the Paterson-Stockmeyer algorithm requires $O(\sqrt{p-3})$ non-scalar multiplications.
  Furthermore, the preprocessing phase of this algorithm computes $z^2, z^4, \dots, z^{2k}$ and $z^{4k}, z^{8k}, \dots, z^{2^r k}$ with $2k(2^r-1) = p-3$.
  We can use these powers to compute the leading term in $m$ steps, namely
  \begin{align*}
    z^2 z^{2k} z^{4k} \cdots z^{2^r k} = z^{2 + 2k(2^r-1)} = z^{2 + p - 3} = z^{p-1}.
  \end{align*}
  Since $r \in O(\log (p/(2k)))$, the asymptotic complexity of evaluating~(\ref{eq:univariate_circuit}) is $O(\sqrt{p-3})$ non-scalar multiplications.
  
  \begin{remark}
    A careful reader can notice that the leading term of~(\ref{eq:univariate_circuit}) is equal to $\chi_p(x-y)$, which is the heaviest part of the equality circuit $\EQ_\S$.
    Thus, we can get $\EQ_\S$ almost for free (at the cost of one $\SubPlain$) after evaluating $\LT_\S$, thus saving $O(\log (p-1))$ non-scalar multiplications.

    This feature of our circuit allows to compute all the equality operations while comparing large integers using the less-than function $\LT$ from~(\ref{eq:general_lex_order}).
    This saves $O((d'-1)(k-1) \log (p-1))$ homomorphic multiplications, thus leading to a better running time.
  \end{remark}

\subsection{Min/max function}

  Given the less-than function $\LT_\S$ defined on a set $\S$, one can compute the minimum of two elements $x$ and $y$ in the following generic way
  \begin{align*}
    \min_\S(x,y) &= x \cdot \LT_\S(x,y) + y \cdot (1 - \LT_\S(x,y)) \\
    &= y + (x-y) \cdot \LT_\S(x,y)
  \end{align*}
  Notice that the difference $x - y$ naturally emerges in this expression, thus hinting that the univariate circuit from~(\ref{eq:univariate_circuit}) might be useful here.
  Indeed, by replacing $x - y$ with an indeterminate $z$ we obtain the univariate polynomial representation of the minimum function 
  \begin{align*}
    \min_\S(x,y) & = y + z \cdot \left(\frac{p+1}{2} z^{p-1} + \sum_{i=1, \text{odd}}^{p-2} z^i \sum_{a=1}^{\frac{p-1}{2}} a^{p-1-i}\right) \\
    & = y + \frac{p+1}{2} z + \sum_{i=1}^{\frac{p-1}{2}} z^{2i} \sum_{a=1}^{\frac{p-1}{2}} a^{p-2i} \\
    & = \frac{p+1}{2} (x+y) + \sum_{i=1}^{\frac{p-1}{2}} z^{2i} \sum_{a=1}^{\frac{p-1}{2}} a^{p-2i} \\
    & = \frac{p+1}{2} (x+y) + g(z^2)
  \end{align*}
  where $g(X)$ is a polynomial of degree $(p-1)/2$. 
  As a result, $\min(x,y)$ can be computed with $O(\sqrt{p-1})$ non-scalar multiplications via the Paterson-Stockmeyer algorithm.

  Following the above reasoning, the maximum function can be computed as follows
  \begin{align*}
    \max_\S(x,y) &= \frac{p+1}{2} (x+y) - \sum_{i=1}^{\frac{p-1}{2}} z^{2i} \sum_{a=1}^{\frac{p-1}{2}} a^{p-2i}.
  \end{align*}
  \begin{remark}
    Maximum and minimum functions are a basic building block in the neural network design.
    For example, one of the most popular activation functions in neural networks is the rectifier, or ReLU.
    The above formula for $\max_\S$ yields the following simple polynomial for the ReLU function
    \begin{align*}
      \ReLU_\S(x) = \max_\S(x,0) &= \frac{p+1}{2} x - \sum_{i=1}^{\frac{p-1}{2}} x^{2i} \sum_{a=1}^{\frac{p-1}{2}} a^{p-2i}.
    \end{align*}
  \end{remark}

% \subsection{Lexicographic order}\label{subsec:lexicographic_order}
%   Let $\vx=(x_0,x_1,\ldots,x_{\ell-1})$ and $\vy=(y_0,y_1,\ldots,y_{\ell-1}) \in \F_\fieldcard^\ell$ for the lexicographical order $<$ defined by the choice of $\S$:
%   \begin{align*}
%     \vx < \vy \Leftrightarrow \exists i\in[0,\ell-1] \text{ such that } x_i < y_i \text{ and } \forall j > i ~~ x_j = y_j\,.
%   \end{align*}
%   This order induces a function $\LT_{\F_\fieldcard}(\vx, \vy)$ that returns $1$ if $\vx < \vy$ and $0$ otherwise. 
%   As done in~\cite{TLWRK20}, we can employ $\EQ_{\F_\fieldcard}$ and $\LT_{\F_\fieldcard}$ to compute $\LT_{\F_\fieldcard}(\vx, \vy)$ as follows
%   \begin{align*}
%     \LT_{\F_\fieldcard}(\vx, \vy) = \sum_{i=0}^{\ell-2} \LT_{\F_\fieldcard}(x_{i}, y_{i}) \prod_{j=i+1}^{\ell-1} \EQ_{\F_\fieldcard}(x_{j}, y_{j}) + \LT_{\F_\fieldcard}(x_{\ell-1}, y_{\ell-1}).
%   \end{align*}
%   Notice that the multiplicative depth of this function solely depends on the products of the equality functions.
%   In fact, these products compare subvectors $\vx_i = (x_i, x_{i+1},\dots,x_{\ell-1})$ and $\vy_i = (y_i, y_{i+1},\dots,y_{\ell-1})$ for $i \in [1,\ell-1]$.
%   Thus, we can rewrite $\LT_{\F_\fieldcard}$ as
%   \begin{align}\label{eq:general_lex_order}
%     \LT_{\F_\fieldcard}(\vx, \vy) = \sum_{i=0}^{\ell-2} \LT_{\F_\fieldcard}(x_{i}, y_{i}) \EQ_{\F_\fieldcard}(\vx_{i+1}, \vy_{i+1}) + \LT_{\F_\fieldcard}(x_{\ell-1}, y_{\ell-1})
%   \end{align}
%   with the equality function $\EQ_{\F_\fieldcard}(\vx_{i+1}, \vy_{i+1})$ that returns $1$ if $\vx_{i+1} = \vy_{i+1}$ and $0$ otherwise.
%   As shown in~\todo{cite our work}, this function can be realized by a constant-depth circuit in the following way.
%   \begin{align}\label{eq:rand_eq_circuit}
%     \EQ_{\F_\fieldcard, e}(\vx_{i+1},\vy_{i+1}) = 1 - \princhar_{\fieldcard^e}\left(\sum_{j=i+1}^{\ell-1} r_j (x_j - y_j)\right)
%   \end{align}
%   where $r_j$ are uniformly random elements from $\F_{\fieldcard^e}$.
%   This circuit is false-biased with error probability $\fieldcard^{-e}$.
%   We can compute all $\EQ_{\F_\fieldcard, e}(\vx_{i+1},\vy_{i+1})$ using the same number of multiplications as for the single equality using Algorithm~\ref{alg:vector_equalities_circuit}.\todo{What is the complexity?}
%   \begin{algorithm}[t]
%     \KwIn{
%     $\ct_\vx$ -- a ciphertext encrypting $\vx \in \F^\ell_\fieldcard$,
%     $\ct_\vy$ -- a ciphertext encrypting $\vy \in \F^\ell_\fieldcard$.
%     }
%     \KwOut{$\ct$ -- a ciphertext containing the output of $\EQ_{\F_\fieldcard, e}(\vx_{i+1},\vy_{i+1})$ in the $i$th SIMD slot}
%     $\ct_1 \leftarrow \Shift(\ct_\vx, 1)$ // removes the value $x_0$ and shifts $\vx$ to the left\\
%     $\ct_2 \leftarrow \Shift(\ct_\vy, 1)$ // removes the value $y_0$ and shifts $\vy$ to the left\\
%     $\ct \leftarrow \Sub(\ct_1, \ct_2)$ // $(x_i - y_i), i \in [1,\ell-1]$\\
%     $r_0,r_1,\dots,r_{\ell-1} \leftarrow \udist(\F_{\fieldcard^e})$ //\todo{define uniform distribution}\\ 
%     $\pt_r \leftarrow \pt(r_0,r_1,\dots,r_{\ell-1})$\\
%     $\ct \leftarrow \ct \cdot \pt_r$ // $r_i(x_i - y_i), i \in [1,\ell-1]$\\
%     $k \leftarrow 1$// $\sum_{j=i}^{\ell-1} r_i(x_i - y_i), i \in [1,\ell-1]$\\
%     \While{$k < \ell-1$}{
%       $\ct_{tmp} \leftarrow \Shift(\ct, k)$\\
%       $\ct \leftarrow \ct + \ct_{tmp}$\\
%       $k \leftarrow 2k$\\
%     }
%     $\ct \leftarrow \Power(\ct, \fieldcard^e-1)$ //$\princhar_{\fieldcard^e}\left(\sum_{j=i}^{\ell-1} r_i(x_i - y_i)\right), i \in [1,\ell-1]$ \\
%     $\ct \leftarrow 1 - \ct$\\
%     \textbf{Return} $\ct$.
%     \caption{Homomorphic circuit computing $\EQ_{\F_\fieldcard, e}(\vx_{i+1},\vy_{i+1})$ in parallel.}\label{alg:vector_equalities_circuit}
%   \end{algorithm}
  
%   Let us go back to the lexicographic order in equation~(\ref{eq:general_lex_order}).
%   Assume that we have a ciphertext $\ct_{\LT}$ containing $\LT_{\F_\fieldcard}(x_i, y_i)$ in the $i$th slot (\todo{elaborate on that}) and the output $\ct_{\EQ}$ of Algorithm~\ref{alg:vector_equalities_circuit}.
%   We set the $\ell-1$ slot of $\ct_{\EQ}$ to $1$ and multiply it by $\ct_{\LT}$.
%   The resulting ciphertext contains all the products $\LT_{\F_\fieldcard}(x_{i}, y_{i}) \EQ_{\F_\fieldcard}(\vx_{i+1}, \vy_{i+1})$ for any $i \in [0,\ell-2]$ and $\LT_{\F_\fieldcard}(x_{\ell-1}, y_{\ell-1})$, which can be summed in the same way as in the while circuit of Algorithm~\ref{alg:vector_equalities_circuit}.
%   As a result, we obtain a ciphertext with the output of $\LT_{\F_\fieldcard}(\vx, \vy)$ in the first SIMD slot.

%   \subsubsection{Comparing large integers.}
%   Since any integer $z$ can be represented in base $\fieldcard$ as $z = \sum_{i=0}^{\ell-1} z_i q^i$, we can encode $z$ into SIMD slots as a vector $(z_0,z_1,\dots,z_{\ell-1}) \in \F_{\fieldcard}^\ell$.
%   As shown above, we can compare such vectors using the lexicographic order function in~(\ref{eq:general_lex_order}) and thus compare integers larger than $\fieldcard$.

%   \todo{Incorporate the following into the complexity analysis.} 
%   We can obtain ciphertexts encrypting $(x,x,\ldots, x)$ and $(y,y,\ldots, y)$ with $2\log_2(q-1)\texttt{Rot}$ and $2\log_2(q-1)\texttt{Add}$. 

%   From there we can obtain encryptions of $(x,x-1,\ldots, x-q-2)$ and $(y-1,y-2,\ldots, y-q-1)$ with $2\texttt{Add}$. We can apply $f$ to these vectors in parallel with a depth of $\log (p-1) + \log d$ for a cost of $2(\log (p-1) + wt(p-1) + d - 2) \texttt{Mult}$. This can be minimized by choosing $p = 2^d + 1$ for some $d$.

%   With $2$ more $\texttt{Sub}$ we obtain encryptions of $(1-f(x), 1-f(x-1), \ldots, 1-f(x-q-2))$ and $(1-f(y-1), 1-f(y-2), \ldots, 1-f(y-q-1))$. From there we can get an encryption of the different partial sums in $\log_2(q-1)\texttt{Rot}$, $\log_2(q-1)\texttt{Select}$ and $\log_2(q-1)\texttt{Add}$. The algorithm works as follow:

%   \begin{center}
%     \begin{tikzpicture}[scale=0.8]
%       \newcommand\y{-4};
      
%       \draw (0,2) -- (16,2);
%       \draw (0,0) -- (16,0);    
%       \foreach \x in {0,...,8}{
%         \draw (2*\x,0) -- (2*\x,2);
%       }
%       \foreach \x in {0,...,7}{
%         \node at (2*\x+1,1) {$x_\x$};
%       }

%       \draw (0,2+\y) -- (16,2+\y);
%       \draw (0,\y) -- (16,\y);    
%       \foreach \x in {0,...,8}{
%         \draw (2*\x,0+\y) -- (2*\x,2+\y);
%       }

%       \node at (1,1+\y) {$x_0 + x_1$};
%       \node at (3,1+\y) {$x_1 + x_2$};
%       \node at (5,1+\y) {$x_2 + x_3$};
%       \node at (7,1+\y) {$x_3 + x_4$};
%       \node at (9,1+\y) {$x_4 + x_5$};
%       \node at (11,1+\y) {$x_5 + x_6$};
%       \node at (13,1+\y) {$x_6 + x_7$};
%       \node at (15,1+\y) {$x_7$};

%       %%%%%%%%%%%%%%%%%%%%%%%%%%%%%%%%%%%%
      
%       \draw (0,2+2*\y) -- (16,2+2*\y);
%       \draw (0,0+2*\y) -- (16,0+2*\y);    
%       \foreach \x in {0,...,8}{
%         \draw (2*\x,0+2*\y) -- (2*\x,2+2*\y);
%       }

%       \node at (1,1+2*\y) {$\begin{array}{c}
%           x_0 + x_1  \\
%           + x_2 + x_3
%         \end{array}$};
%       \node at (3,1+2*\y) {$\begin{array}{c}
%           x_1 + x_2  \\
%           + x_3 + x_4
%         \end{array}$};
%       \node at (5,1+2*\y) {$\begin{array}{c}
%           x_2 + x_3  \\
%           + x_4 + x_5
%         \end{array}$};
%       \node at (7,1+2*\y) {$\begin{array}{c}
%           x_3 + x_4  \\
%           + x_5 + x_6
%         \end{array}$};
%       \node at (9,1+2*\y) {$\begin{array}{c}
%           x_4 + x_5  \\
%           +x_6 + x_7
%         \end{array}$};
%       \node at (11,1+2*\y) {$\begin{array}{c}
%           x_5 + x_6  \\
%           + x_7
%         \end{array}$};
%       \node at (13,1+2*\y) {$x_6 + x_7$};
%       \node at (15,1+2*\y) {$x_7$};

%       %%%%%%%%%%%%%%%%%%%%%%%%%%%%%%%%%%%%%%%%%%
      
%       \draw (0,2+3*\y) -- (16,2+3*\y);
%       \draw (0,0+3*\y) -- (16,0+3*\y);    
%       \foreach \x in {0,...,8}{
%         \draw (2*\x,0+3*\y) -- (2*\x,2+3*\y);
%       }

%       \node at (1,1+3*\y) {$\begin{array}{c}
%           x_0 + x_1  \\
%                               + x_2 + x_3 \\
%                               + x_4 + x_5 \\
%                               + x_6 + x_7
%         \end{array}$};
%       \node at (3,1+3*\y) {$\begin{array}{c}
%           x_1 + x_2  \\
%                               + x_3 + x_4 \\
%                               + x_5 + x_6 \\
%                               + x_7 \\
%         \end{array}$};
%       \node at (5,1+3*\y) {$\begin{array}{c}
%           x_2 + x_3  \\
%                               + x_4 + x_5 \\
%                               + x_6 + x_7
%         \end{array}$};
%       \node at (7,1+3*\y) {$\begin{array}{c}
%           x_3 + x_4  \\
%                               + x_5 + x_6 \\
%                               +x_7
%         \end{array}$};
%       \node at (9,1+3*\y) {$\begin{array}{c}
%           x_4 + x_5  \\
%           +x_6 + x_7
%         \end{array}$};
%       \node at (11,1+3*\y) {$\begin{array}{c}
%           x_5 + x_6  \\
%           + x_7
%         \end{array}$};
%       \node at (13,1+3*\y) {$x_6 + x_7$};
%       \node at (15,1+3*\y) {$x_7$};

%       \foreach \x in {2,...,8}
%       \draw[->,thick = 2pt] (2*\x-0.1-1,2) to[bend right] (2*\x-3+0.1,2);

%       \foreach \x in {3,...,8}
%       \draw[->,thick = 2pt] (2*\x-0.1-1,2+\y) to[bend right] (2*\x-5+0.1,2+\y);

%       \foreach \x in {5,...,8}
%       \draw[->,thick = 2pt] (2*\x-0.1-1,2+2*\y) to[bend right] (2*\x-9+0.1,2+2*\y);

%   \end{tikzpicture}
%   \end{center}

%   From there we just have to compute the scalar product of the two vectors with $1\texttt{Mult} + \log_2(q-1)\texttt{Rot} + \log_2(q-1)\texttt{Add}$. \newline

%   So overall we can compute $<$ over $\mathbb{F}_q$ for $q = p^d$ in:

%   \begin{itemize}
%   \item $4\log_2 (q-1) \texttt{Rot}$
%   \item $(4\log_2 (q-1) + 4) \texttt{Add}$
%   \item $(2(\log (p-1) + wt(p-1) + d - 2) + 1) \texttt{Mult}$
%   \item $(\log_2 (q-1) + 1) \texttt{Select}$
%   \end{itemize}


%%% Local Variables:
%%% mode: latex
%%% TeX-master: "main"
%%% End:



\section{Applications}
\label{sec:applications}
\subsection{Sorting}
\label{subsec:sorting}

	To demonstrate the efficiency of our algorithm, we applied it to a popular computational task that demands multiple comparisons, sorting.
	The best sorting algorithm in terms of running time is the direct sorting algorithm due to~\cite{CDSS15}.
	For a given array $\va = (a_0,\dots,a_{n-1})$, this algorithm computes a \emph{comparison matrix} $\mL$ whose entries are defined by
	\begin{align*}
		\mL_{ij} =
		\begin{cases}
			\LT(a_i, a_j) & \text{ if } i < j, \\
			0 & \text{ if } i = j, \\
			1 - \LT(a_j, a_i) & \text{ if } i > j.
		\end{cases}
	\end{align*}
	It is easy to see that the Hamming weight of the $i$th row of $\mL$ is unique and equal to the array index of $a_i$ after sorting $\va$ in the descending order.
	For example, the zero weight indicates that there are no elements of $\va$ bigger than $a_i$; thus, $a_i$ is the first value of $\va$ after sorting, or the maximum element of $\va$.

	To arrange $\va$ in the right order, we extract its elements by comparing the current array index $i$ with the Hamming weights of the comparison matrix rows ($\wt(\mL[j])$).  
	\begin{align*}
		a'_i = \sum_{j=0}^{n-1} \EQ(i,\wt\left(\mL[j]\right)) \cdot a_j.
	\end{align*}
	Since each row has a unique Hamming weight, there exist only one $a_j$ that will be assigned to $a'_i$.

	\begin{remark}
		Since the matrix $\mL$ is defined by $n(n-1)/2$ elements, it can be costly to keep it in memory.
		Instead we can compute the Hamming weights of its rows by iteratively computing one comparison $\LT(a_i, a_j)$ with $i < j$ at a time.
		To achieve this, we create an array of size $n$ initialized with zeros that eventually will store the Hamming weights.
		Then, we add the outcome of $\LT(a_i, a_j)$ to the $i$th element of this array and the result of $1-\LT(a_i, a_j)$ to the $j$th element.
		In this approach, only $n$ elements of the Hamming weight array are being kept in RAM.
	\end{remark}

	The direct sorting algorithm requires $n(n-1)/2$ less-than operations and $n^2$ equality operations.
	While computing equalities, we can reduce the total number of ciphertext-ciphertext multiplications if $n$ is large enough.
	Recall that $\EQ(i,\wt\left(\mL[j]\right)) = 1 - (i-\wt\left(\mL[j]\right))^{p-1}$, which implies that $\EQ$ needs $M = \log_2(p-1) + \wt(p-1) - 1$ ciphertext-ciphertext multiplications.
	Hence, to compute $\EQ(i,\wt\left(\mL[j]\right))$ for all $i \in [0,n-1]$, we should perform $n M$ multiplications.
	Using Lemma~\ref{lem:difference_to_p-1}, we can rewrite $\EQ(x,y) = 1 - \sum_{k=0}^{p-1} i^k \cdot \wt\left(\mL[j]\right)^{p-1-k}$.
	If we precompute the powers $\wt\left(\mL[j]\right)^{p-1-k}$, then we need only $p-2$ multiplications to compute all the equalities $\EQ(i,\wt\left(\mL[j]\right))$.
	Hence, if $p-2 < nM$, or $n > (p-2)/M$, the latter approach results in a smaller number of ciphertext-ciphertext multiplications.
	One can argue that the latter approach introduces $p-1$ plaintext-ciphertext multiplications ($i^k \cdot \wt\left(\mL[j]\right)^{p-1-k}$) and $p-2$ additions. 
	However, these operations are much faster in practice than ciphertext-ciphertext multiplication such that the gain from reduced ciphertext-ciphertext multiplications becomes dominant. 
	
	The main advantage of direct sorting is that its multiplicative depth is independent of the array length, namely $d = d(\LT) + d(\EQ) + d(\wt) + 1$.
	This allows to avoid large encryption parameters and costly bootstrapping operations.
	We can further reduce this depth by computing the Hamming weight modulo a plaintext modulus $p$ that is equal or larger than the length of an array $n$.

\subsection{Min/max of an array}
\label{sec:min/max}

	Another application of our comparison algorithm is concerned with the minimum (or maximum) function of an array.
	To find the minimum of an input array, at least $n-1$ calls of the pairwise minimum function are required~\cite[Chapter 9]{CLR09}, which can be achieved by \emph{the tournament method}.
	In this method, one reduces the size of the input array in $\ceil{\log n}$ iterations.
	In each iteration, the input array is divided into pairs. 
	If the array length is odd, one element is stashed for the next iteration. 
	Then, the maximum of each pair is removed from the array.
	The algorithm stops when only one element is left; this is the minimum of the input array, see Figure~\ref{fig:minimum_tournament}. 
	\begin{figure}
		\centering
		\begin{tikzpicture}
			% first stage
			\draw[black, thin] (0,0) rectangle (1,1);
			\draw[black, thin] (0,1.5) rectangle (1,2.5);
			\draw[black, thin] (0,3.0) rectangle (1,4);
			\draw[black, thin] (0,4.5) rectangle (1,5.5);

			\node at (0.5, 5.0) {$a_0$};
			\node at (0.5, 3.5) {$a_1$};
			\node at (0.5, 2.0) {$a_2$};
			\node at (0.5, 0.5) {$a_3$};

			% second stage
			\draw[black, thin] (1.5,0.75) rectangle (2.5,1.75);
			\draw[black, thin] (1.5,3.75) rectangle (2.5,4.75);

			\node at (2.0, 1.25) {$\min$};
			\node at (2.0, 4.25) {$\min$};

			\draw[black, thin, ->] (1,0.5) -- (2,0.5) -- (2,0.75);
			\draw[black, thin, ->] (1,2) -- (2,2) -- (2,1.75);

			\draw[black, thin, ->] (1,3.5) -- (2,3.5) -- (2,3.75);
			\draw[black, thin, ->] (1,5) -- (2,5) -- (2,4.75);

			% third stage
			\draw[black, thin] (3.0,2.25) rectangle (4.0,3.25);

			\node at (3.5, 2.75) {$\min$};

			\draw[black, thin, ->] (2.5,1.25) -- (3.5,1.25) -- (3.5,2.25);
			\draw[black, thin, ->] (2.5,4.25) -- (3.5,4.25) -- (3.5,3.25);

			% final outcome
			\draw[black, thin, ->] (4.0,2.75) -- (5.0,2.75);
			\node at (6.5, 2.75) {$\min(a_0,a_1,a_2,a_3)$};

		\end{tikzpicture}
		\caption{The tournament method of finding the minimum of an array. In each stage, the array elements are divided into pairs. Only minimum of a pair go to the next stage.}
		\label{fig:minimum_tournament}
	\end{figure}
	Unfortunately, the tournament method has a big multiplicative complexity, namely $\ceil{\log n} d(\min)$, which enforces us to use either impractical encryption parameters or a slow bootstrapping function.

	To reduce the depth, we can combine the tournament method and direct sorting.
	First, we perform $\ell$ iterations of the tournament algorithm, which leaves us with an array $\va' = (a_0, \dots, a_{n'-1})$ of length $n' = \ceil{n/2^\ell}$ containing the minimum.
	Then, we can find the minimum by computing the comparison table as in direct sorting and extracting an element with the Hamming weight of its row equal to $n'-1$, i.e.
	\begin{align*}
		\min(\va') = \sum_{j=0}^{n'-1} \EQ(n'-1,\wt\left(\mL[j]\right)) \cdot a'_j.
	\end{align*}
	Unfortunately, this approach involves quadratic number of comparison operations ($\LT$) due to the computation of the comparison table.
	However, the depth is reduced to $\ell \cdot d(\min) + d(\LT) + d(\EQ) + 1$, which allows us to set reasonable encryption parameters and to avoid bootstrapping, see Section~\ref{sec:impl-results}.

\section{Implementation Results}
\label{sec:impl-results}
\begin{table}
  \centering
  \begin{tabular}{||c|c|c||ccccc||}
    \hline
    $~m~$ & $~n~$ & $~\delta_\mathcal{R}~$ & $~t~$ & $~d~$ & $~\ell~$ & $~s~$ & $~\log_2(t^d)~$ \\
    \hline
    $22,383$ & $14,904$ & $7n$ & $5$  & $18$ & $828$  & $1$ &  $41.7$ \\
    $25,623$ & $15,552$ & $175n$ & $2$ & $36$ & $432$ & $2$ & $36.0$ \\
    $31,726$ & $15,288$ & $59n$ & $3$ & $28$ & $546$ & $1$ & $44.3$ \\
    \hline
    $31,697$ & $30,576$ & $59n$ & $3$ & $28$ & $1092$ & $1$ & $44.3$ \\
    $65,536$ & $32,768$ & $2n$ & $2$ & $32$ & $2,048$ & $1$ & $32.0$ \\
    $93,006$ & $30,996$ & $7n$ & $5$ & $18$ & $1,722$ & $1$ & $41.7$ \\
    \hline
                                                   
  \end{tabular}
  \caption{Potential parameters}
  \label{tab:params}
\end{table}

\begin{itemize}
\item $m$: cyclotomic order;
\item $n$: degree of the cyclotomic;
\item $\delta_\mathcal{R}$: expansion factor (upper-bound);
\item $t$: plaintext modulus;
\item $d$: degree of the factors of $\phi_m$ modulo $t$;
\item $\ell$: numbers of factors of of $\phi_m$ modulo $t$;
\item $s$: number of generators for $(\Z/m\Z)^*/<t>$;
\item $\log_2(t^d)$: size of the slots/inverse of the probability of failure for one equality check.
\end{itemize}

%%% Local Variables:
%%% mode: latex
%%% TeX-master: "main"
%%% End:


% \section{Old ideas}
% The homomorphic circuit for $\LT_{\F_\fieldcard}(x,y)$ is presented in Algorithm~\ref{alg:basic_less_than_circuit}.
Unlike the prior works (\todo{cite}) that evaluate this circuit as a multivariate polynomial, we exploit its representation in~(\ref{eq:less_than_function}) to parallelize computations in the SIMD manner. 
\begin{algorithm}[t]
  \KwIn{
  $\ct_x$ -- a ciphertext encrypting $x \in \F_\fieldcard$ in the first SIMD slot and $0$ in other slots,
  $\ct_y$ -- a ciphertext encrypting $y \in \F_\fieldcard$ in the first SIMD slot and $0$ in other slots,
  $\S = \{a_0,a_1,\dots,a_{\fieldcard-1}\}$ -- a complete set of representatives of $\F_\fieldcard$.
  }
  \KwOut{$\ct$ -- a ciphertext encrypting $1$ in the first SIMD slot if $x < y$, otherwise $0$. Other slots contain $0$.}
  $\ct_1 \leftarrow \Replicate(\ct_x, \fieldcard - 1)$//\todo{Define replicate}\\
  $\ct_2 \leftarrow \Replicate(\ct_y, \fieldcard - 1)$\\
  $\pt_1 \leftarrow \pt(a_0, a_1, \dots, a_{\fieldcard-2})$//\todo{define this operation}\\
  $\pt_2 \leftarrow \pt(a_1, a_2, \dots, a_{\fieldcard-1})$\\
  $\ct_1 \leftarrow \ct_1 - \pt_1$//\todo{define operations between ciphertexts and plaintexts}\\
  $\ct_2 \leftarrow \ct_2 - \pt_2$\\
  $\ct_1 \leftarrow \ct_1^{\fieldcard-1}$//$\princhar_\fieldcard(x-a)$, \todo{define this operation}\\
  $\ct_2 \leftarrow \ct_2^{\fieldcard-1}$//$\princhar_\fieldcard(y-b)$\\
  $\ct_1 \leftarrow 1 - \ct_1$// define operations between constants and ciphertexts\\
  $\ct_2 \leftarrow 1 - \ct_2$\\
  $k \leftarrow 1$// compute all running sums $\sum_{b \in \S, a < b} (1-\princhar_\fieldcard(x - b))$ \\
  \While{$k < \fieldcard-1$}{
    $\ct_{tmp} \leftarrow \Shift(\ct_2, k)$//\todo{define shift}\\
    $\ct_2 \leftarrow \ct_2 + \ct_{tmp}$\\
    $k \leftarrow 2k$\\
  }
  $\ct \leftarrow \ct_1 \cdot \ct_2$\\
  $k \leftarrow 1$\\
  \While{$k < \fieldcard-1$}{
    $\ct_{tmp} \leftarrow \Shift(\ct, k)$\\
    $\ct \leftarrow \ct + \ct_{tmp}$\\
    $k \leftarrow 2k$\\
  }
  $\ct \leftarrow \Select(\ct, \{1\})$//\todo{define select}\\
  \textbf{Return} $\ct$.
  \caption{Homomorphic circuit of $\LT_{\F_\fieldcard}$.}\label{alg:basic_less_than_circuit}
\end{algorithm}
\todo{Make a separate algorithm for running sums (while circuit above)}
\todo{Write down the complexity of Algorithm~\ref{alg:basic_less_than_circuit}, compare with prior works}

Here are potential cyclotomic polynomials to use for a failure probability smaller $\varepsilon < 2^{-32}$.

\begin{table}
  \setlength{\tabcolsep}{1em}
  \centering
  \begin{tabular}{||ccc||cccccc||}
    \hline
    $~m~$ & $~n~$ & $~\delta_\mathcal{R}~$ & $~p~$ & $~d~$ & depth & $~\ell~$ & $~s~$ & $~\varepsilon~$ \\
    \hline
    $22,383$ & $14,904$ & $5n$ & $5$  & $18$ & $8$ & $828$  & $1$ &  $2^{-41.7}$ \\
    $25,623$ & $15,552$ & $118n$ & $2$ & $36$ & $7$ & $432$ & $2$ & $2^{-36.0}$ \\
    $31,132$ & $15,120$ & $85n$ & $7$ & $12$ & $8$ & $1,260$ & $2$ & $2^{-33.6}$ \\
    $31,726$ & $15,288$ & $57n$ & $3$ & $28$ & $7$ & $546$ & $1$ & $2^{-44.3}$ \\
    $35,552$ & $16,000$ & $21n$ & $17$ & $10$ & $9$ & $1,600$ & $3$ & $2^{-40.8}$ \\
    \hline
    $31,697$ & $30,576$ & $57n$ & $3$ & $28$ & $7$ & $1,092$ & $1$ & $2^{-44.3}$ \\
    $48,771$ & $32,508$ & $5n$ & $2$ & $42$ & $7$ & $774$ & $1$ & $2^{-42.0}$ \\
    $57,368$ & $28,000$ & $141n$ & $17$ & $10$ & $9$ & $2,800$ & $3$ & $2^{-40.8}$ \\
    $72,400$ & $28,800$ & $9n$ & $7$ & $12$ & $8$ & $2,400$ & $3$ & $2^{-33.6}$ \\
    $93,006$ & $30,996$ & $5n$ & $5$ & $18$ & $8$ & $1,722$ & $1$ & $2^{-41.7}$ \\
    \hline
                                                   
  \end{tabular}
  \caption{Potential parameters}
  \label{tab:params}
\end{table}

\begin{itemize}
\item $m$: cyclotomic order;
\item $n$: degree of the cyclotomic;
\item $\delta_\mathcal{R}$: expansion factor (upper-bound);
\item $p$: plaintext modulus;
\item $d$: degree of the factors of $\phi_m$ modulo $p$;
\item depth: $\lceil \log_2(p-1) \rceil + \lceil \log_2(d)\rceil + 1$;
\item $\ell$: number of factors of of $\phi_m$ modulo $p$;
\item $s$: number of generators for $(\Z/m\Z)^\times/<p>$;
\item $\varepsilon$: approximative probability of failure for one equality check.
\end{itemize}

\remove{
\subsection{Min/Max function on vectors}

  \begin{align*}
    \min_\S(\vx,\vy) = \vx \cdot \LT_\S(\vx,\vy) + \vy \cdot (1 - \LT_\S(\vx,\vy)).
  \end{align*}

  \begin{align*}
    \LT_{\F_\fieldcard}(\vx, \vy) = \sum_{i=0}^{\ell-2} \LT_{\F_\fieldcard}(x_{i}, y_{i}) \prod_{j=i+1}^{\ell-1} \EQ_{\F_\fieldcard}(x_{j}, y_{j}) + \LT_{\F_\fieldcard}(x_{\ell-1}, y_{\ell-1}).
  \end{align*}

  \begin{align*}
    (x-y) \cdot \EQ_{\F_p}(x,y) = 0
  \end{align*}
  
  \begin{align*}
    \min_\S(\vx,\vy)_k &= x_k \cdot \LT_\S(\vx,\vy) + y_k \cdot (1 - \LT_\S(\vx,\vy)) \\
    &= y_k + (x_k - y_k) \cdot \LT_\S(\vx,\vy)) \\
    &= y_k + (x_k - y_k) \cdot \left(\sum_{i=0}^{\ell-1} \LT_{\F_\fieldcard}(x_{i}, y_{i}) \prod_{j=i+1}^{\ell-1} \EQ_{\F_\fieldcard}(x_{j}, y_{j})\right) \\
    &= y_k + (x_k - y_k) \cdot \left(\sum_{i=0}^{k-1} \LT_{\F_\fieldcard}(x_{i}, y_{i}) \prod_{j=i+1}^{\ell-1} \EQ_{\F_\fieldcard}(x_{j}, y_{j})\right) \\
    &+ (x_k - y_k) \cdot \left(\sum_{i=k}^{\ell-1} \LT_{\F_\fieldcard}(x_{i}, y_{i}) \prod_{j=i+1}^{\ell-1} \EQ_{\F_\fieldcard}(x_{j}, y_{j})\right) \\
    &= y_k + (x_k - y_k) \cdot \left(\sum_{i=k}^{\ell-1} \LT_{\F_\fieldcard}(x_{i}, y_{i}) \prod_{j=i+1}^{\ell-1} \EQ_{\F_\fieldcard}(x_{j}, y_{j})\right)
  \end{align*}
  }

\section{Conclusion}
In this work, we constucted more efficient homomorphic circuits of comparison operations for the BGV and BFV FHE schemes.
Our results are based on structural properties of comparison functions over finite fields.
We proved that less-than functions of two input variables $x$ and $y$ can be represented either by bivariate polynomials or univariate polynomials (in variable $z = x - y$) with multiple zero coefficients, which simplifies computation.
Moreover, our computation of the univariate less-than functions yields the output of the equality function almost for free, which allowed us to speed up the lexicographic order of vectors over finite fields and thus comparison of large integers encoded by these vectors.

The implementation of our circuits in HElib is faster than the state-of-the-art work~\cite{TLWRK20} by more than a factor of 3.
Furthermore, the running time of our circuits is comparable to implementations of comparison algorithms in TFHE, which is believed to be the most efficient FHE scheme for non-arithmetic homomorphic computations.
As a side contribution, we applied our comparison algorithms to the tasks of sorting and array minimum search.
In both cases we achieved the best running time present in the literature.
For instance, our sorting algorithm can sort 4676 batches of 64 32-bit integers in about 25 hours, which results in an amortized time equal to about 19 seconds per batch.
Our minimum circuit can find minimal elements of 2610 batches of 64 32-bit integers in approximately 7 hours; the amortized time is 9.5 seconds per batch.

We hope that this work will draw attention to the study of arithmetic circuits over finite fields representing non-arithmetic functions over integers, thus leading to practically efficient homomorphic implementations of useful algorithms.


\bibliographystyle{plain}
\bibliography{abbrev3,biblio}

\appendix
\section{Proof of Theorem \ref{thm:less_than_total_degree}}
\label{app:proof_thm_less_than_total_degree}
To prove this statement we will need the following Lemma.

  \begin{lemma}\label{lem:difference_to_p-1}
    For all $(a,b)\in\Z^2$ we have:
    \[
      (a - b)^{p-1} = \sum_{i=0}^{p-1} a^i b^{p-1-i} \mod p.
    \]
  \end{lemma}
  \begin{proof}
    Using the binomial theorem we obtain
    \[
      (a - b)^{p-1} = \sum_{i=0}^{p-1} \binom{p-1}{i} a^i (-b)^{p-1-i}.
    \]
    Computing the binomial coefficient modulo $p$
    \begin{align*}
      \binom{p-1}{i} & = \frac{(p-1)!}{i! (p-1-i)!} \\
      & = \frac{(p-1)(p-2)\dots(i+1)}{1 \cdot 2 \dots (p-(1+i))} \\
      &= (-1)^{p-1-i} \mod p\,,
    \end{align*}
    we prove the lemma.
  \end{proof}

  \begin{lemma}\label{lem:sum_poly}
    Let $P(X)$ be a polynomial of degree $d$ less than $p-1$.
    For any prime number $p > 2$, it holds
    \[
      \sum_{a=0}^{p-1} P(a) = 0 \mod p.
    \]
  \end{lemma}
  \begin{proof}
    It is enough to prove that the sum $\sum_{a=0}^{p-1} a^n = 0 \mod p$ for any $0 \leq n < p-1$. The case $n=0$ is straightforward, now let us assume $n>0$. Let $g$ be a primitive element of $\F_p$.
    Since $p > 2$, we have $g \ne 1$.
    Thus, we can rewrite the above sum as follows.
    \begin{align}
      \sum_{i=1}^{p-1} g^{in} = \frac{g^{pn} - g^n}{g^n - 1}.
    \end{align}
    Since $g^{pn} \equiv g^n \mod p$, the sum turns into zero modulo $p$.
  \end{proof}
  Now, We have all the ingredients to prove Theorem \ref{thm:less_than_total_degree}.
  \begin{proof}[Theorem \ref{thm:less_than_total_degree}]
    Assume that all computations are done modulo $p$.
    Using Lemma~\ref{lem:difference_to_p-1}, we obtain that $P_{\LT_\S}(X,Y)$ is equal to
    \begin{align*}
      \sum_{a = 0}^{p-2} \left(1-\sum_{i=0}^{p-1} X^i a^{p-1-i}\right) \sum_{b=a+1}^{p-1} \left(1-\sum_{j=0}^{p-1} Y^j b^{p-1-j}\right)
    \end{align*}
    Let us expand this expression distributively.
    \begin{align*}
      \sum_{a = 0}^{p-2} \sum_{b=a+1}^{p-1} 1 - \sum_{i=0}^{p-1} X^i a^{p-1-i} - \sum_{i=0}^{p-1} Y^i b^{p-1-i} \\
      + \sum_{i=0}^{p-1} \sum_{j=0}^{p-1} X^i Y^j a^{p-1-i} b^{p-1-j}.
    \end{align*}
    Let us separately compute polynomial coefficients.
    The constant term is equal to
    \begin{align*}
      \sum_{a = 0}^{p-2} \sum_{b=a+1}^{p-1} 1 - a^{p-1} - b^{p-1} + a^{p-1} b^{p-1} \\
      = \sum_{a = 0}^{p-2} \sum_{b=a+1}^{p-1} 1 - a^{p-1} - 1 + a^{p-1} = 0 
    \end{align*}
    The coefficients by $X^i$ with $i > 0$ can be computed as follows
    \begin{align*}
      \sum_{a = 0}^{p-2} \sum_{b=a+1}^{p-1} \left(-a^{p-1-i}\right) + a^{p-1-i} = 0.
    \end{align*}
    Next, we compute the coefficients by $Y^i$ with $i > 0$.
    \begin{align*}
      -\sum_{a = 0}^{p-2} &\sum_{b=a+1}^{p-1} b^{p-1-i} - a^{p-1} b^{p-1-i} \\
      &= -\sum_{b=1}^{p-1} b^{p-1-i} - \sum_{a = 1}^{p-2} \sum_{b=a+1}^{p-1} b^{p-1-i} - b^{p-1-i} \\
      &= -\sum_{b=1}^{p-1} b^{p-1-i}.
    \end{align*}
    If $i = p-1$, this sum is equal to $1$.
    According to Lemma~\ref{lem:sum_poly}, it is $0$ if $i \ne 0 \mod (p-1)$.
    
    To compute coefficients by $X^i Y^j$ with $i, j > 0$, we will use Faulhaber's formula below
    \begin{align*}
      \sum_{k=1}^n k^e = \frac{1}{e+1} \sum_{i=1}^{e+1} (-1)^{\delta_{ie}} \binom{e+1}{i} B_{e+1-i} \cdot n^i\,,
    \end{align*}
    where $\delta_{ie}$ is the Kronecker delta and $B_{i}$ is the $i$th Bernoulli number.
    This implies that there exist a polynomial $P(X) \in \F_p[X]$ of degree $e+1$ such that
    \begin{align}\label{eq:faulhaber}
      \sum_{k=1}^n k^e = P(n).
    \end{align}
    Note that $P(0) = 0$.
    The coefficient by $X^i Y^j$ for some positive $i$ and $j$ is equal to
    \begin{align*}
      \sum_{a = 0}^{p-2} \sum_{b=a+1}^{p-1} a^{p-1-i} b^{p-1-j} = \sum_{b = 1}^{p-1} b^{p-1-j} \sum_{a=0}^{b-1} a^{p-1-i}.
    \end{align*}
    According to~(\ref{eq:faulhaber}), it follows that there exist a polynomial $P(X)$ of degree $p-i$ such that $\sum_{a=0}^{b-1} a^{p-1-i} = P(b)$.
    Since $Q(X) = X^{p-1-j} P(X)$ has degree $2p-1-i-j$, Lemma~\ref{lem:sum_poly} implies that if $i+j > p$, then
    \begin{align*}
      \sum_{b = 1}^{p-1} b^{p-1-j} \sum_{a=0}^{b-1} a^{p-1-i} = \sum_{b=1}^{p-1} Q(b) = 0.
    \end{align*}
    Thus, the total degree of $P_{\LT_\S}(X,Y)$ is at most $p$.

    In addition, we consider the case when $i = j$.
    Let us consider the following sum
    \begin{align*}
      \sum_{a=0}^{p-1} a^{p-1-i} \sum_{b=0}^{p-1} b^{p-1-i} = 0.
    \end{align*}
    We can rewrite it as follows
    \begin{align*}
      \sum_{a=0}^{p-1} & a^{p-1-i} \sum_{b=0}^{p-1} b^{p-1-i} \\
      &= 2\sum_{a=0}^{p-2} a^{p-1-i} \sum_{b=a+1}^{p-1} b^{p-1-i} + \sum_{a=0}^{p-1} a^{2(p-1-i)}.
    \end{align*}
    If $i \ne (p-1)/2$, then the last sum is zero.
    Thus, the coefficient by $X^i Y^i$ is equal to $\sum_{a=0}^{p-2} a^{p-1-i} \sum_{b=a+1}^{p-1} b^{p-1-i} = 0$.
    If $i = (p-1)/2$, then following the above argument, this coefficient is equal to $-(p-1)/2$.
  \end{proof}

\section{Decomposition of $f(X,Y)$ for $3\leq p \leq 7$}
\label{app:decomposition-f}
Let $Z = Y(X-Y)$.

\textbf{p=3.}
$$f(X,Y) = 2.$$

Since the polynomial $f(X,Y)$ is constant, it can be computed without any homomorphic multiplication.

\textbf{p=5.}
$$f(X,Y) = 4X^2 + 4X + Z.$$

Two homomorphic multiplications ($\Mul$) are needed to compute $X^2$ and $Z$.

\textbf{p=7.}
$$f(X,Y) = 1 + 4X(X+1) + 6[X(X+1)]^2 + (X^2+3X)Z + 6Z^2$$

In this case, four homomorphic multiplications ($\Mul$) are needed (indicated in bold) when rewritten as follows

$$f(X,Y) = 1 + 2(\bm{X}^2+X)\bm{\cdot}[2 + 3(X^2+X)] + \bm{Z}\bm{\cdot}[(X^2+3X) + 6Z].$$

\section{Proof of Theorem~\ref{th:univariate}}
\label{app:proof-lem-univariate}
   Let $Z = X-Y$.
    Thus we can rewrite $Q_{\LT_\S}(X,Y)$ as the univariate function $\chi_{\F_p^-}$, namely
    \begin{align*}
      Q_{\LT_\S}(X,Y) = \chi_{\F_p^-}(Z) = \sum_{a=-\frac{p-1}{2}}^{-1} 1 - (Z - a)^{p-1}.
    \end{align*}
    Thanks to Lemma~\ref{lem:difference_to_p-1}, we can expand $(Z-a)^{p-1}$ and obtain
    \begin{align*}
      \sum_{a=-\frac{p-1}{2}}^{-1} 1 - \sum_{i=0}^{p-1} Z^i a^{p-1-i}
      = \sum_{i=1}^{p-1} Z^i \sum_{a=-\frac{p-1}{2}}^{-1} (-a^{p-1-i}).
    \end{align*}
    If $i$ is even, then the $i$th coefficient is equal to
    \begin{align*}
      -\sum_{a=-\frac{p-1}{2}}^{-1} a^{p-1-i} &= -\sum_{a=1}^{\frac{p-1}{2}} a^{p-1-i} = -\frac{\sum_{a=-\frac{p-1}{2}}^{\frac{p-1}{2}} a^{p-1-i}}{2}.
    \end{align*}
    This coefficient is equal to $0$ for any even $0<i < p-1$ as $\sum_{a=0}^{p-1} a^{p-1-i} = 0$ in this case.
    The $(p-1)$-th coefficient is equal to $-(p-1)/2 = (p+1)/2 \bmod p$.
    If $i$ is odd, then we can rewrite the $i$th coefficient in the following way
    \begin{align*}
      -\sum_{a=-\frac{p-1}{2}}^{-1} a^{p-1-i} &= \sum_{a=1}^{\frac{p-1}{2}} a^{p-1-i},
    \end{align*}
    which finishes the proof.
 

%%%Local Variables:
%%% mode: latex
%%% TeX-master: "main"
%%% End:


\end{document}

%%% Local Variables:
%%% mode: latex
%%% TeX-master: t
%%% End:
