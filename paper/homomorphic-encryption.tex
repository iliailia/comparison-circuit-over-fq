We are interested in leveled HE schemes that support SIMD operations on their plaintexts.
Such schemes include $\FV$~\cite{FV12} and $\BGV$~\cite{BGV12}, have the ring $\intring_p$ as plaintext space for some $p \geq 2$.
We refer to such schemes as \emph{SIMD-schemes}.
The general framework of such schemes is outlined below.

\subsection{Basic setup}

Let $\lambda$ be the security level of an HE scheme.
Let $L$ be the maximal multiplicative depth of homomorphic circuits we want to evaluate.
Let $d$ be the order of the plaintext modulus $p$ modulo the order $m$ of $\intring$.
Assume that the plaintext space $\intring_p$ has $\slots$ SIMD slots, i.e. $\intring_p \cong \F^\ell_{p^d}$.
% For a vector $\va \in \F^k_{p^d}$, we denote the plaintext encoding of $\va$ by $\pt(\va)$. 
The basic part of a SIMD-scheme consists of key generation, encryption and decryption algorithms.

$\KeyGen(1^\lambda, 1^L) \rightarrow (\sk, \pk)$. Given $\lambda$ and $L$, this function generates the secret key $\sk$ and the public key $\pk$.
Note that the public key contains special key-switching keys that help to transform ciphertexts encrypted under other secret keys back to ciphertexts encrypted under $\sk$. 

$\Encrypt(\pt \in \intring_t, \pk) \rightarrow \ct$. The encryption algorithm takes a plaintext $\pt$ and the public key $\pk$ and outputs a ciphertext $\ct$.

$\Decrypt(\ct, \sk) \rightarrow \pt$. The decryption algorithm takes a ciphertext $\ct$ and the secret key $\sk$ and returns a plaintext $\pt$.
For freshly encrypted ciphertexts, the decryption correctness means that $\Decrypt(\Encrypt(\pt, \pk), \sk) = \pt$. 

\subsection{Arithmetic operations}

Homomorphic arithmetic operations are addition and multiplication.

$\Add(\ct_1, \ct_2) \rightarrow \ct$. The addition algorithm takes two input ciphertexts $\ct_1$ and $\ct_2$ encrypting plaintexts $\pt_1$ and $\pt_2$ respectively.
It outputs a ciphertext $\ct$ that encrypts the sum of these plaintexts in the ring $\intring_p$.
%It implies that homomorphic addition sums respective SIMD slots of $\pt_1$ and $\pt_2$.

$\AddPlain(\ct_1, \pt_2) \rightarrow \ct$. This algorithm takes a ciphertext $\ct_1$ encrypting a plaintext $\pt_1$ and a plaintext $\pt_2$.
It outputs a ciphertext $\ct$ that encrypts $\pt_1 + \pt_2$.
%As for the $\Add$ algorithm, $\AddPlain$ sums respective SIMD slots of $\pt_1$ and $\pt_2$.

$\Mul(\ct_1, \ct_2) \rightarrow \ct$. Given two input ciphertext $\ct_1$ and $\ct_2$ encrypting plaintext $\pt_1$ and $\pt_2$ respectively, the multiplication algorithm outputs a ciphertext $\ct$ that encrypts the plaintext product $\pt_1 \cdot \pt_2$.
%As a result, homomorphic multiplication multiplies respective SIMD slots of $\pt_1$ and $\pt_2$.

$\MulPlain(\ct_1, \pt_2) \rightarrow \ct$. Given a ciphertext $\ct_1$ encrypting plaintext $\pt_1$ and a plaintext $\pt_2$, this algorithm outputs a ciphertext $\ct$ that encrypts the plaintext product $\pt_1 \cdot \pt_2$.
%As a result, homomorphic multiplication multiplies respective SIMD slots of $\pt_1$ and $\pt_2$.

Using the above operations as building blocks, one can design homomorphic subtraction algorithms.

$\Sub(\ct_1, \ct_2) = \Add(\ct_1, \MulPlain(\ct_2, \pt(-\1))) \rightarrow \ct$. The subtraction algorithm returns a ciphertext $\ct$ that encrypts the difference of two plaintext messages $\pt_1 - \pt_2$ encrypted by $\ct_1$ and $\ct_2$, respectively.

$\SubPlain(\ct_1, \pt_2) = \AddPlain(\ct_1, \pt_2 \cdot \pt(-\1)) \rightarrow \ct$. This algorithm returns a ciphertext $\ct$ that encrypts $\pt_1 - \pt_2$ where $\pt_1$ is encrypted by $\ct_1$.
We can also change the order of arguments such that $\SubPlain(\pt_1, \ct_2)$ returns a ciphertext $\ct$ encrypting $\pt_1 - \pt_2$. 

As shown in Section~\ref{subsec:crt}, the projection map $\proj_I$ can select the SIMD slots indexed by a set $I \subseteq \{0,\dots,\slots-1\}$ and set the rest to zero.
This functionality is homomorphically realized by the $\Select$ function.

$\Select(\ct, I) = \MulPlain(\ct, \pt(\1_I)) \rightarrow \ct'$ where $\1_I$ is a vector having $1$'s in the coordinates indexed by a set $I$ and zeros everywhere else.
Given a ciphertext $\ct$ encrypting SIMD slots $\vm = (m_0,m_1,\dots,m_{\slots-1})$ and a set $I$, this function returns a ciphertext $\ct'$ that encrypts $\vm' = (m'_0,\dots,m'_{\slots-1})$ such that $m'_i = m_i$ if $i \in I$ and $m'_i = 0$ otherwise.

\subsection{Special operations}\label{subsec:special_operations}

One can also homomorphically permute the SIMD slots of a given ciphertext and act on them with the Frobenius automorphism. 

$\Rotate(\ct, i) \rightarrow \ct'$ with $i \in [0, \slots-1]$. Given a ciphertext $\ct$ encrypting SIMD slots 
$$\vm = (m_0,m_1,\dots,m_{\slots-1}),$$
the rotation algorithm returns a ciphertext $\ct'$ that encrypts the cyclic shift of $\vm$ by $i$ positions, namely $(m_i,m_{(i+1) \mod \slots},\dots,m_{(i-1) \mod \slots})$.

$\Frob(\ct, i) \rightarrow \ct'$ with $i\in[0,d-1]$. Given a ciphertext $\ct$ encrypting SIMD slots $\vm$ as above, the Frobenius algorithm returns a ciphertext $\ct'$ that encrypts a Frobenius map action on $\vm$, namely $(m^{p^i}_0,m^{p^i}_1,\dots,m^{p^i}_{\slots-1})$.

As discussed in Section~\ref{subsec:crt}, the $\Frob$ and $\Mul$ operations can be combined to compute the principal character $\princhar$, which turns non-zero values of SIMD slots into $1$ and leaves slots with zeros unchanged.

$\IsNonZero(\ct) \rightarrow \ct'$. Given a ciphertext $\ct$ encrypting SIMD slots $\vm = (m_0,m_1,\dots,m_{\slots-1})$, this function returns a ciphertext $\ct'$ that encrypts:
\[
  (\princhar(m_0), \princhar(m_1), \dots, \princhar(m_{\slots-1})).
\]
% Kim et al.~\cite{TDSC:KLLW16} showed that one can employ the Frobenius map to decrease the multiplicative depth of $\IsNonZero$.
Recall that $\princhar(m) = m^{p^d-1} = \prod_{i=0}^{d-1} (m^{p-1})^{p^i}$ as shown in~(\ref{eq:exp_frob}).
The multiplicative depth of $x^{p-1}$ is equal to $\ceil{\log_2 (p-1)}$.
The multiplicative depth of $x^{p^i}$ is zero as it can be done by the $\Frob$ operation.
In total, $d-1$ $\Frob$ operations are needed to compute $\princhar(m)$.
As a result, the total multiplicative depth of $\IsNonZero$ is
\begin{align}\label{eq:nonzero_depth}
  \ceil{\log_2 (p-1)} + \ceil{\log_2 d}.
\end{align}
Using general exponentiation by squaring, $x^{p-1}$ requires $\floor{\log_2 (p-1)} + \wt(p-1) - 1$ field multiplications.
Since $d-1$ field multiplications are needed to compute $\prod_{i=0}^{d-1} (x^{p-1})^{p^i}$, the total number of multiplications to compute $\princhar(m)$ is:
\begin{align}\label{eq:nonzero_width}
  \floor{\log_2 (p-1)} + \wt(p-1) + d - 2.
\end{align} 

\subsection{Cost of homomorphic operations}\label{subsec:cost}
Note that every homomorphic ciphertext contains a special component called \emph{noise} that is removed during decryption.
However, the decryption function can deal only with noise of small enough magnitude; otherwise, this function fails.
This noise bound is defined by encryption parameters in a way that larger parameters result in a larger bound.
The ciphertext noise increases after every homomorphic operation and, therefore, approaches its maximal possible bound.
It implies that to reduce encryption parameters one needs to avoid homomorphic operations that significantly increase the noise.
Therefore, while designing homomorphic circuits, we need to take into account not only the running time of homomorphic operations but also their effect on the noise.  

Table~\ref{table:he_operations} summarizes the running time and the noise cost of the aforementioned homomorphic operations.
Similar to~\cite{C:HalSho14}, we divide the operations into expensive, moderate and cheap.
The expensive operations dominate the cost of a homomorphic circuit.
The moderate operations are less important, but if there are many of them in a circuit, their total cost can dominate the total cost.
The cheap operations are the least important and can be omitted in the cost analysis.

It is worth to note that there are two multiplication functions $\Mul$ (ciphertext-ciphertext multiplication) and $\MulPlain$ (ciphertext-plaintext multiplication).
Since $\Mul$ is much more expensive than $\MulPlain$, the multiplicative depth of a homomorphic circuit is usually calculated with relation to the number of $\Mul$'s.

\begin{table}[t!]
  \centering
  \begin{tabular*}{.45\textwidth}{@{\extracolsep{\fill} } c c c }
    \toprule
    Operation	& Time			& Noise \\
    \midrule
    $\Add$		& cheap			& cheap 	\\
    $\AddPlain$	& cheap			& cheap \\
    $\Mul$		& expensive		& expensive 	\\
    $\MulPlain$	& cheap			& moderate 	\\
    $\Sub$		& cheap			& cheap  \\
    $\SubPlain$	& cheap			& cheap  \\
    $\Select$	& cheap			& moderate \\
    $\Rotate$ 	 & expensive	& moderate \\
    $\Frob$		 & expensive	& cheap \\
    $\IsNonZero$ & expensive    & expensive \\
    \bottomrule
  \end{tabular*}
  \caption{The cost of homomorphic operations with relation to running time and noise growth.}
  \label{table:he_operations}
\end{table}

%%% Local Variables:
%%% mode: latex
%%% TeX-master: "main"
%%% End:
